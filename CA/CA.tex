\documentclass[11pt]{article}
\usepackage{amssymb}
\usepackage{latexsym}
\usepackage{amsmath}
\usepackage{amsthm}
\usepackage{mathtools}
\usepackage{natbib}
\usepackage{tikz-cd}
\usepackage{enumitem} 
\usepackage{hyperref}
\hypersetup{
    colorlinks,
    citecolor=blue,
    filecolor=blue,
    linkcolor=blue,
    urlcolor=blue
}
\newtheorem{thm}{Theorem}[section]
\newtheorem{prop}[thm]{Proposition}
\newtheorem{lemma}[thm]{Lemma}
\newtheorem{cor}[thm]{Corollary}
\newtheorem{dfn}[thm]{Definition}
\newtheorem{axiom}[thm]{Axiom}
\newtheorem{rmk}[thm]{Remark}
\newtheorem{ex}[thm]{Example}
\newtheorem{question}[thm]{Question}
\newtheorem{problem}[thm]{Problem}
\newtheorem{dfn/thm}[thm]{Definition/Theorem}
\renewcommand{\baselinestretch}{1.05}
\newcommand{\reals}{\mathbb R}
\newcommand{\cplx}{\mathbb C}
\newcommand{\intg}{\mathbb Z}
\newcommand{\bbk}{\mathbb K}
\newcommand{\bbf}{\mathbb F}
\newcommand{\ratl}{\mathbb Q}
\newcommand{\torus}{\mathbb T}
\newcommand{\sca}{{\mathfrak a}}
\newcommand{\scb}{{\mathfrak b}}
\newcommand{\scc}{{\mathfrak c}}
\newcommand{\scm}{{\mathfrak m}}
\newcommand{\scn}{{\mathfrak n}}
\newcommand{\scp}{{\mathfrak p}}
\newcommand{\scq}{\mathfrak q}
\newcommand{\frakg}{{\mathfrak g}}
\newcommand{\frakd}{{\mathfrak d}}
\newcommand{\calf}{{\cal F}}
\newcommand{\calg}{{\cal G}}
\newcommand{\cala}{{\cal A}}
\newcommand{\calb}{{\cal B}}
\newcommand{\calc}{{\cal C}}
\newcommand{\cale}{{\cal E}}
\newcommand{\call}{{\cal L}}
\newcommand{\caln}{{\cal N}}
\newcommand{\calo}{{\cal O}}
\newcommand{\calr}{{\cal R}}
\newcommand{\mathbold}{\bf}
\newcommand{\cinf}{C^{\infty}}
\newcommand{\row}[2]{#1_1,\dots ,#1_{#2}}
\newcommand{\dbyd}[2]{{\partial #1\over\partial #2}}
\newcommand{\Space}{{\bf Space}}
\newcommand{\alg}{{\mathbold Alg}}
\newcommand{\notsubset}{\not \subset}
\newcommand{\notsupset}{\not \supset}
\newcommand{\pois}{{\mathbold Pois}}
\newcommand{\pitilde}{\tilde{\pi}}
\newcommand{\rta}{\rightarrow}
\newcommand{\Lrta}{\Longrightarrow}
\newcommand{\lrta}{\longrightarrow}
\newcommand{\llrta}{\longleftrightarrow}
\newcommand{\Llta}{\Longleftarrow}
\newcommand{\Llrta}{\Longleftrightarrow}
\newcommand{\lgl}{\langle}
\newcommand{\rgl}{\rangle}
\newcommand{\inj}{\hookrightarrow}
\newcommand{\downmapsto}{\rotatebox[origin=c]{-90}{$\scriptstyle\mapsto$}\mkern2mu}
\renewcommand{\qedsymbol}{$\square$}
\bibliographystyle{plain}
\title{\bf Personal Notes for Commutative Algebra by P. Nelson}
\author{Texed by Lin-Da Xiao} %\thanks{Research partially supported by NSF Grant DMS-96-25122 and the Miller Institute for Basic Research in Science.}
\begin{document}
\maketitle
\tableofcontents
\newpage
\section*{About the Course:}
The course website is https://metaphor.ethz.ch/x/2017/hs/401-3132-00L/. 

The topic includes
\begin{itemize}
\item 
Basics about rings, ideals and modules
\item 
Localization
\item 
Primary decomposition
\item 
Integral dependence and valuations
\item 
Noetherian rings
\item 
Completions
\item 
Basic dimension theory
\end{itemize}

Prerequisite:

Rings, homomorphism, ideals, quotient rings, zero divisors, prime/maximal ideals, fields.

Convention: 
Ring, we mean a commutative ring with identity.
 
In particular for a ring homomorphism $f:R\rightarrow S$. We have $f(1_R)=1_S$.
 
Remark:
we allow 1=0 but then R=0.
Caution, by definition $1\neq 0$ in a field .
\section{Rings, ideals, radicals}
\subsection{Lecture 1. Motivation and Basics by Paul Steinmann}
In differential geometry, we have the theorem of level sets:
\begin{thm}
Let $f:\reals^n \rightarrow \reals$.
If $0 \in \reals^n$ is a regular value of $f$ then $f^{-1}(0)$ is a submanifold.
\end{thm} 
 
In algebraic geometry, we look at $f^{-1}(0)$ for polynomial $f$.
More precisely, fix an algebraic-closed field $\bbk$ and an integer $n>0$, consider the ring $R:=\bbk[x_1,...,x_n]$.
 
Def: For a subset $S\subset R$ we define the \textbf{affine algebraic variety} by
\begin{equation}
V(S):=\{x\in \bbk^n| \forall f \in S,\ f(x)=0\subset \bbk^n\}
\end{equation}
\begin{rmk}
With the affine algebraic varieties defined above, we have:
\begin{itemize}
\item $V(\emptyset)=\bbk^n$
\item $ V(\{1\})=\emptyset$
\item   For an non empty collection of subsets $(S_i)_{i\in I}$ $S_i\subset R$ we have 
\begin{equation*}
    \cap_{i\in I}V(S_i)=V(\cup_{i\in I} S_i)
\end{equation*}
\item $S$ and $S'$ are subsets in $R$
\begin{equation*}
    V(S)\cup V(S')=V(\{fg| f\in S, g\in S'\})
\end{equation*}
\end{itemize}
as a consequence, 
    $(V(S))_{S\subset R}$ form the closed sets of a topology on $\bbk^n$ called \textbf{Zariski topology}.
\end{rmk}
\begin{ex}
n=2 , $R=\bbk[X_1,X_2]$

 $V(\{X_1\})$ is the $X_2$ axis in $\bbk^2$

 $V(\{X_2-X_1^2\})$ is the parabola in $\bbk^2$
\end{ex}

\begin{dfn}
Conversely for all subset $X \subset \bbk^n$, consider 
          $$ I(X):=\{f\in R | \forall x\in X: f(x)=0\}\subset R.$$
\end{dfn}
\begin{rmk}
Fact: For S in R and X subset in $\bbk^n$, we have, 
\begin{itemize}
\item
 $S\subset I(V(S))$
\item
 $X \subset V(I(X))$
\item
 For $S\subset S' \subset$ in $R$, we have $V(S)\supset V(S')$
\item
 For $X \subset X'\subset \bbk^n$, we have $ I(X)\supset I(X')$
\item
 $I(X) \subset R$ is  an ideal.
\end{itemize}
\end{rmk}

\begin{dfn}
The \textbf{radical of an ideal}   $a\subset R$ is 
 $rad(\sca):=\{a\in R| \exists n\geq1\  s.t.\  a^n \in \sca\} \subset R$
An ideal $\sca \subset R$ with $rad(\sca)$ is  called \textbf{radical}.
\end{dfn}

\begin{rmk}
Fact, 
for every ideal $\sca \subset R$ we have  $\sca\subset rad(\sca)$.

 $rad(\sca)$ is an ideal, proof in exercise.

For $X\subset \bbk^n$ the ideal $I(X)$ is radical. 

\end{rmk}

\begin{thm}
(The Hilbert's Nullstelensatz)
For any ideal $\sca \subset R $ we have
$$I(V(\sca))=rad(\sca).$$
\end{thm}

An important consequence of the theorem:

the maps $V$ and $I$ induce the one to one correspondence between
$$
\{\text{radical ideals in the polynomial ring}\}\Longleftrightarrow                   \{\text{affine algebraic varieties}\}
$$
and this correspondence inverse the inclusion.

\begin{ex}
For any point $x=(x_1,...,x_n) \in \bbk^n$ the ideal 
$$
I({x})=\scm_x:=(X_1-x_1,...,X_n-x_n)
$$
is maximal.
\begin{proof}

 If not, then there exists an ideal $\sca \subset R$
s.t. 
\begin{equation*}
R\supsetneq \sca \supsetneq \scm_x,
\end{equation*}
but then by the Nullstellensatz,

\begin{equation*}
\emptyset\subsetneq V(\sca) \subsetneq V(\scm_x)=\{x\},
\end{equation*}
which makes the contradiction.
\end{proof}
\end{ex}

Weak Nullstellensatz the ideals $m_x$ is are precisely the maximal ideals of $\bbk[x_1,...,x_n]$, where $\bbk$ needs to be algebraically closed
\begin{ex}
$\bbk=\reals, n=1$.
$X^2+1$ is irreducible in $\reals[X]$.
And $\reals[X]/(X^2+1) \cong \cplx$ is maximal.
Consequence, we have a bijection
$$
\{\text{max ideals of R polynomial ring }\bbk[X_1,...,X_n]\}
\Longleftrightarrow\{\text{Points in } \bbk^n\}
$$
\end{ex}

Let $A$ be a ring.
Remember 

    An element $a\in A$ is \textbf{nilpotent} if there $\exists n>1\in \intg$ s.t. $a^n=0$.

    An element $a\in A$ is a \textbf{zero divisor} if there is an element $b\in A, b\neq 0$ s.t. $ab=0$.

Fact: every nilpotent element is a zero divisor but not conversely.

\begin{ex}
     take $(0,1)\in A\times A$ then $(0,1)\cdot(1,0) =(0,0)$
\end{ex}

\begin{dfn}
 The ideal $N:rad((0))$ is called  the \textbf{nil radical} of $A$.
\end{dfn}

Then we have:
\begin{enumerate}
\item $\caln$ is the set of all nilpotent elements  of $A$
\item $A/\caln$ has no nilpotent elements.
\end{enumerate}
\begin{proof}
1. From definitions.
2. Let $x\in A$ s.t. $\bar{x}\in A/\caln$ is nilpotent.
 Let $n>0$ s.t. $\bar{x}^n=0$ then $x^n \in \caln$
Thus there exists $k>0$ s.t. $(x^n)^k=0$ hence $x^{nk}=0$
, $x\in \caln$.
\end{proof}

\begin{prop}
The nil radical of A is the intersection of all prime ideals of A.
\end{prop}
\begin{proof}
Denote by $\caln'$ the intersection of all prime ideals of $A$.
For any nilpotent element $f\in A$ with $n>0$ s.t. $f^n=0$,
We have $f^n\in \scp$ for every prime ideal $\scp$. Hence $f\in \scp$ We conclude $f\in \caln'$
Conversely, suppose $f\in A$ is not nilpotent 
Define $\Sigma:=\{ \sca \subset A\text{ ideals} | \forall n>0: f^n\notin \sca\}$
We will apply Zorn's lemma. 
We have 

    1.  $(0)\in \Sigma$, so $\Sigma$ is nonempty,

    2.  $\Sigma$ is partially ordered by inclusion.

    3.  For any chain $(a_i)_{i\in I}\subset \Sigma$, the set $\sca :=\cup_{i\in I}a_i$ is an ideal and 

    for all $n>0$, 
    we have $f^n \notin \sca$, hence $\sca \in \Sigma$.
By Zorn's lemma we conclude that there is a maximal element $\scp\in \Sigma$.
We show that $\scp$ is a prime ideal.

For any $x,y \notin \scp$, consider the ideals
$\scp+(x), \scp+(y)$. They strictly contain $\scp$ and are thus not in $\Sigma$.
Let $n,m>0$ s.t. $f^n\in (x), f^m\in \scp+(y)$.
We conclude that $f^{n+m}\in \scp+(xy)$, so $\scp+(xy) \notin \Sigma$.
Hence $xy \notin \scp$, which means, $\scp $ is a prime ideal so $f \notin \caln'$.
\end{proof}

Remember let $f:A \rightarrow B$ be a ring morphism. And $\scp\subset B$ a prime ideal . 
Then $f^{-1}(\scp)$ is  a prime ideal of $A$.
Caution: Not true for maximal ideals in general. 
 
\begin{prop}
Let $\sca\subset A$ be an ideal, $\pi : A\rightarrow A/\sca$
There is a one to one correspondence between ideals of $A/\sca$ and ideals in $A$ which contain
$\sca$ via $\scc=\pi^{-1}(\scb)$
\end{prop}

\begin{cor}
Let $\sca\subset  A $ be an ideal, then $rad(\sca)$ is the intersection of all prime ideals which contain $\sca$.
\end{cor}
\begin{proof}
consider the homomorphism $\pi :A\rightarrow A/\sca$
Then $rad(\sca)=\pi^{-1}(\caln_{A/\sca})$.
By the above proposition $\caln_{A/\sca}$ is the intersection of all prime ideals of $A/\sca$.
 By the correspondence we conclude the statement.
\end{proof}
\begin{dfn}
 The \textbf{Jacobson Radical} $\calr$ of $A$ is the intersection of all maximal ideals in $A$.
\end{dfn}
\begin{prop}\label{prop:criterion_jacobson_radical}
We have 
$x\in \calr\Longleftrightarrow \forall y \in A: 1-xy $ is a unit.
\end{prop}
\begin{proof}
``$\Longrightarrow$'' let $x\in \calr$ and $y\in A$ s.t. $1-xy$ is not a unit. Then $1-xy\in \scm$ for some maximal ideal 
$\scm \subset A$. But $x\in \calr\subset \scm,$ hence $1\in \scm$  contradiction.

``$\Longleftarrow$'' let $x  \notin \calr$ then $x \notin \scm$ for some maximal ideal $\scm\subset A$.
Since $\scm$ is maximal we conclude that $(x)+\scm=A$. Hence there exists $y\in A,\   u\in \scm$ s.t. $xy+u=1$.
We conclude that $1-xy \in \scm $, so in particular , $1-xy$ is not a unit.
 \end{proof}
\subsection{Lecture 2. local rings, coprime ideals, ideal quotients by Paul Steinmann}
\begin{dfn}
A ring $A$ is called a \textbf{local ring} if $A$  admits precisely one maximal ideal;
\end{dfn}

\begin{ex}
\ 
\begin{itemize}
\item
Every field is a local ring with maximal ideal $\scm=0$, because every nonzero element is a unit.
\item
$\bbk[[X]]$ is the ring of formal power series over a field $\bbk$, it has a unique maximal ideal $(X)$. One can check that every element with nonzero constant term is invertible. i.e. $(a_0(1-g))^{-1}=a_0^{-1}(1+g+g^2+...)$
\end{itemize}
\end{ex}
 
\begin{prop}\ 
\begin{itemize}
\item 
Let $A$ be a ring and $\scm\neq (1)$ is an ideal of $A$ s.t. every $x\in A-\scm$  is a unit of $A$, then $A$ is a local ring with  maximal ideal $\scm$.
\item
Let $A$ be ring and $\scm\subset A$ is a maximal ideal s.t. any element of $1+\scm=\{1+a|a\in \scm \}$ is a unit in $A$. Then $A$ is a local ring.
\end{itemize}
\end{prop}

\begin{proof}
For first part, every proper ideal consists of non-units, hence is contained in $\scm$. In other words, an element is a unit iff it is not contained in any maximal ideal.
For the second part, let $x\in A-\scm$. Since $\scm$ is maximal , we have $(x)+\scm=(1)$ , hence , $\exists y\in A, t\in \scm$, s.t. $xy+t=1$, which implies $xy=1-t\in 1+\scm.$ Thus ,$xy$ is a unit which implies that $x$ is a unit, Now use the first part.
\end{proof}

\begin{dfn}
A ring $A$ is called \textbf{semilocal} if $A$ admits finitely many  maximal ideals.
\end{dfn}

\begin{ex}\ 
\begin{itemize}
\item 
$\intg$ is not semilocal.
\item
Let $m\in \intg$. Then $\intg/(m\intg)$ is a semilocal ring with maximal ideals $ d\intg/m\intg$ for prime number $d|m$.
\item In particular, for $p\in \intg$ prime , $\intg/p \intg$ is local ring.
\end{itemize}
\end{ex}

 Reminder: Let $\sca,\scb \subset A$ be ideals their sum is 
$$
\sca+\scb:=\{a+b|a\in \sca, b\in \scb\},
$$
Which is the smallest ideal containing $\sca\cup\scb$.
Also infinite sums $(\sca_i)_{i\in I}\subset A$ ideals,
$$
\sum_{i\in I }\sca_i :=\left\{\sum_{i\in I}x_i | x_i\in \sca_i  x_i =0 \text{for almost all i}\right\}
$$
And we also have
$$
\sca\cdot \scb\ or\ \sca\scb=\left\{\sum_{i\in I}x_i y_i|x_i \in \sca , y_i\in \scb, \text{all but finitely many terms are }0\right\}.
$$
\begin{dfn}
Two ideals $\sca, \scb\subset A$ are called \textbf{coprime}\footnote{In some literature, it is called \textbf{comaximal}} if $\sca+\scb=(1)$ 
\end{dfn}
\begin{rmk}
If $\sca,\scb\subset A$ are coprime ideals then $\sca\cap \scb =\sca\cdot \scb$.\\
For general ideals $\sca,\scb\subset A$ :
$$
(\sca+\scb)\cdot(\sca\cap \scb)\subset \sca\cdot \scb\subset \sca\cap \scb.
$$
However, for coprime ideals, we also have $\sca\scb \supset \sca\cap\scb$, because $1=a+b$ for $a\in\sca, b\in \scb$, then
$\forall x\in \sca\cap\scb$ we have $x=x\cdot 1=x(a+b)=xa+xb\in \sca\cdot \scb$.
\end{rmk}

\begin{prop}
Let $\sca_1,...,\sca_n \subset A$ be ideals, denote $\varphi: A\rightarrow \prod_{i\in I}^n(A/\sca_{i})$ for the canonical homomorphism.\\
(i) if $\sca_i,\sca_j$ are coprime for $i\neq j$ , then $\prod_{i=1}^n\sca_i=\cap_{i=1}^n\sca_i$.\\
(ii)$\varphi$ is surjective iff $\sca_i,\sca_j$ are coprime for $i\neq  j$.\\
(iii) $\varphi$ is injective iff $\cap_{i=1}^n\sca_i=(0)$.
\end{prop}

\begin{proof}
(iii) Note that $ker \varphi =\cap^n_{i=1}\sca_i$.\\
(i) by induction on $n$.
For $n=2$ it is checked above. Suppose $n>2$ let $\scb:=\prod_{i=1}^{n-1}\sca_i=\cap^{n-1}_{i=1}\sca_i$
Since $\sca_i+\sca_n=(1)$ for $1\leq i\leq n-1$. We have $x_i+y_i=1$ for some $x_i\in \sca_i, y_i\in \sca_n$
Thus $\prod_{i=1}^{n-1}x_i= \prod_{i=1}^{n-1}(1-y_i) \equiv 1 \mod \sca_n$
We conclude that $\sca_n+\scb=(1)$, s.t. 
$$
\prod^n_{i=1}\sca_i=\scb \sca_n=\sca\cap \sca_n=\cap_{i=1}^n\sca_i
$$
(ii) ``$\Longrightarrow$'', Suppose $\varphi $ is surjective. Let $i\neq j$, There exists an element $x\in A$ s.t. $\varphi(x)=(0,...,0,1,0,...,0)$, nonzero only at the $i$-th entry.
Thus $x\equiv  1 \mod \sca_i$ and $x\equiv 0\mod \sca_j$.
So $1=(1-x)+x\in \sca_i+\sca_j$.\\
``$\Longleftarrow$'' We show that for all $k\in \{1,...,n\}$
 there exists  an element $x\in A$ s.t. $\varphi(x)=(0,..0,1,0..0)$, nonzero at the $k$-th entry.
Let $k\in \{1,...,n\}$. For every $j\in \{1,...,n\}\backslash\{k\}$.
We have $\sca_k+\sca_j=(1)$, and thus there are elements $u_j\in\sca_k,v_j\in \sca_j$
s.t. $u_j+v_j=1$.
Define $x:=\prod_{i\neq k}v_i$. Then $x\equiv 0 \mod \sca_j,\ \forall j\neq k$ and $x=\prod_{i\neq k}(1-u_i)\equiv 1\mod \sca_k$.
Hence, $\varphi(x)=(0,..,0,1,0,...,0)$ nonzero in the $k$-th entry.\\
As a result, if each pair $\sca_i,\sca_j$ is coprime, we have 
$$
A/\left(\prod_{i=1}^n\sca_i\right)\cong \prod_{i=1}^n\left(A/\sca_i\right).
$$
\end{proof}

 
\begin{prop}
Le t $\sca,\scb\subset  A$ be ideals s.t. $rad(\sca),rad(\scb)$ are coprime. Then $\sca, \scb$ are coprime.
\end{prop}
\begin{proof}
In fact, we have
$$
 rad(\sca+\scb)=rad(rad(\sca)+rad(\scb))=rad((1))=(1)
$$
Details in the exercise sheet.
\end{proof}

\begin{prop}\ \\
(i) Let $\scp_1,..., \scp_n\subset A$ prime ideals and let $\sca \subset A$ be an ideal which is contained in $\cup_{i=1}^n\scp_i$ then $\sca\subset \scp_j$  for some $j$.\\
(ii)Let $\sca_1,...,\sca_n\subset A$ be ideals and $\scp \subset A$ a prime ideal s.t. $\scp \supset \cap_{i=1}^n\sca_i$.
Then $\scp\supset \sca_i$ for some $i$.
If $\scp=\cap_{i=1}^n\sca_i$,  then $\scp =\sca_i $ for some $i$ .
\end{prop}

 
\begin{proof}
Induction on $n$. For $n=1$, easily checked. For $n > 1$. Assume that $\sca \notsubset \scp_i$ 
for all $1\leq i\leq n$. We show $\sca\notsubset \cup_{i=1}^n\scp_i$.
By induction hypothesis we know that $\forall k,\sca\notsubset \cup_{i\neq k}^n\scp_i $, so there exists $x_k\in \sca$ s.t. $x_k\notin \scp_i,\ \forall i\neq k$.We choose an $x_k$ for each $\scp_k$ in the above manner.
If $x_k\notin \scp_k$ for some $k$ , then we are done.
If not, then $x_k\in \scp_k$ for all $k$. 
Consider $y:=\sum_{k=1}^n\prod_{j\neq k}x_j$. We have $y\in \sca$ and $y \equiv \prod_{j\neq k}x_j \mod \scp_k,\forall k$.
Since $x_j\notin \scp_k$ for $j\neq k$ and $\scp_k$ is a prime ideal, we conclude that $y\notin \scp_k$ for all $k$ hence $\sca\notsubset \cup_{i=1}^n \scp_i$.\\
(ii) Suppose for all $i\in \{1,...,n\}$ we have $\scp\notsupset \sca_i$. Then there 
are $x_i\in\sca_i $ with $x_i\notin \scp$ for all $i$. 
And thus $\prod_{i=1}^n x_i\in \prod_{i=1}^n\sca_i\subset \cap_{i=1}^n\sca_i$. 
Since $\scp$ is a prime ideal $\prod_{i=1}^n x_i\notin \scp$, hence $\scp\notsupset \cap_{i=1}^n \sca_i$.
If $\scp=\cap_{i=1}^n\sca_i\subset \sca_k$ for all $k$, which produce the last part.
\end{proof}

\begin{dfn}
Let $\sca,\scb\subset A$ be two ideals. Their \textbf{ideal quotient} is 
$$
(\sca:\scb):=\{x\in A| x\scb \subset \sca\}.
$$ 
The \textbf{annihilator} of an ideal $\sca\subset A $ is 
$$
Ann(\sca):=\{(0):\sca\}.
$$
\end{dfn}
Notation: For $x\in A$ we write $(a:x):=(a:(x))$.\\
Fact: 
(i)The ideal quotient of two ideals is again an ideal.\\
(ii)The set of zero divisors of $A$ is 
$$
D=\cup_{x\neq 0} Ann(x)=\cup_{x\neq 0}(Ann(x))
$$
\begin{proof}
(i)
(ii) The first equality is just by definition. The the second equality.
$$
D=rad(D)=rad(\cup_{x\neq 0}Ann(x))=\cup_{x\neq 0}rad(Ann(x)),
$$
 where we extend rad to arbitrary subsets.
\end{proof}

Properties: Let $\sca, \scb\subset A$ be ideals \\
(i)$\sca\subset (\sca:\scb)$\\
(ii) $(\sca:\scb)\scb\subset \sca$\\
(iii)$((\sca:\scb):\scc)=(\sca:\scb\cdot\scc)=((\sca:\scc):\scb)$\\
(iv) for ideals $(\sca_i)_{i\in I}\subset A$, $(\cap_{i\in I}\sca_i:\scb)=\cap_{i\in I}(\sca_i:\scb)$\\
(v) for ideals $(\scb_i)_{i\in I}\subset A$ , $(\sca:\sum_{i\in I }\scb_i)=\cap_{i\in I} (\sca:\scb_i)$.\\
 
\begin{dfn}
Let $\sca\subset A$ be an ideal $f: A\rightarrow  B$ a ring homomorphism. We define the \textbf{extension} of $\sca$ by $f$ to be the ideal 
$$
\sca^e:=f_{*}(\sca):=B f(\sca)
$$,
Which is just the ideal in $B$ generated by $f(a)$\\
For an dieal $\scb\subset B$. We define the \textbf{contraction} of $\scb$ via $f$ to be the ideal 
$$
\scb^c:=f^*(\scb):=f^{-1}(\scb)
$$
\end{dfn}

Properties: Let $f: A\rightarrow B$ be a ring homomorphism , $\sca\subset A$ $\scb\subset B$ ideals. Then :\\
(i) $\sca\subset f^*f_*(\sca)=\sca^{e c}$,$\scb\supset f_{*}f^*(\scb)=\scb^{ce}$.\\
(ii) $f^*(\scb)=f^* f_*f^*(\scb), f_*(\sca)=f_* f^*f_*(\sca)$.\\
(iii) Denote by $C$ the set of contracted ideals in $A$ and by $E$ the set of extended ideals in $B$, then 
$$
C=\{\sca\subset A|f^*f_*(\sca)=\sca\},
$$
 
$$
E=\{\scb\subset B| f_* f^*(\scb)=\scb\}.
$$ 
And $f_*: C\rightarrow E$ is a bijection with inverse $f^*$.
\begin{proof}
 For (i), $\sca\subset f^{-1}f(\sca)\subset f^{-1}f_{*}(\sca)=f^{*}f_*(\sca)$. For (ii) $\scb \supset f(f^{-1}(\scb))$ and $\scb$ is an ideal so $\scb \supset f_{*}f^*(\scb)$. Part (iii) is left as an exercise.
\end{proof}
\section{Modules}
\subsection{Lecture 3. Modules, Exact sequences by Professor Kowalski}
Outline of this chapter 
\begin{itemize}
\item Definition examples and Nakayama's Lemma
\item exact sequences , snake lemma
\item tensor products
\item Algebra over a ring
\end{itemize}
Roughly speaking, module is ``vector spaces for rings''. It is closely related to fibre bundles in geometry. For the convention, we still fix commutative ring $\cala$ with unit.
\begin{dfn}
A \textbf{module} $M$ over $\cala$ is an Abelian group with a linear action of $\cala$ on $M$, i.e.
$$
\begin{aligned}
& \cala\times M\rightarrow M\\
& (a,x)\mapsto ax
\end{aligned}
$$
so that 
$$
\begin{aligned}
&a(x+y)=ax+ay\\
&(a+b)x=ax+bx\\
&a(bx)=ab x\\
& 1 x= x
\end{aligned}
$$
\end{dfn}

\begin{ex}
\begin{enumerate}
\item $\{0\}$ is an $\cala$-module
\item if $\cala$ is a field $\cala$-module is just $\cala$-vector space.
\item $I\subset\cala$ ideal; then $I$is an $\cala$-module (a submodule of $\cala$)
\item $\cala=\intg$, an $\cala$-module is an abelian group.
\end{enumerate}
\end{ex}

\begin{dfn}
$M$ and $N$ are $\cala$-modules $f:M\rightarrow N$ is  \textbf{$\cala$-linear} if $f(ax+by)=af(x)+bf(y)$. The set of such $\rho:M\rightarrow N$ is denoted $Hom_{\cala}(M,N)$. It is an $\cala$-module with
$$
(f+g)(x)=f(x)+g(x),
$$
$$
(af)(x)=af(x).
$$
\end{dfn}
If
$Q\overset{h}{\lrta }M\overset{f}{\lrta} N\overset{g}{\lrta} P$, then $g\circ f\in Hom_\cala(M,P)$ and $g\circ(f\circ h)=(g\circ f)\circ h$. Also, $id_M\in Hom_\cala(M,M)$. In other word, $\cala$-module is a category.

\begin{dfn}
$f:M\rta N$ is an \textbf{isomorphism} iff $\exists g:N\lrta M$ s.t. $g\circ f=id_N$ and $f\circ g=id_M$.
\end{dfn}
\begin{rmk}
$Q\rta(h) M\rta(f) N\rta(g) P$, then for any $P$, we get
 $$
 f^*: Hom_\cala(M,P)\rta Hom_\cala(M,P)
 $$
 $$
g\mapsto g\circ f
 $$
 and 
 $$
 f_*: Hom_\cala(Q,M)\rta Hom_\cala(Q,N)
 $$
 $$
h\mapsto  f\circ h
 $$
They are $\cala$-linear, because for example
$$
\begin{aligned}
(f^*(a h+b g))(x)&=((a h+b g)\circ f) (x)\\
&= (a h+b g) (f(x))\\
&=a h(f(x))+bg(f(x))\\
&= (a f^*(h)+b f^* (g))(x).
\end{aligned}
$$
\end{rmk}

\begin{rmk}
Suppose $M$ is an $\cala$-module and $N\subset M$ as submodule, then $M/N$ has the structure of $\cala$-module such that the canonical projection $\pi: M\rta M/N$ is $\cala$-linear. $a(x+N)=ax+N$ is well defined because $a N\subset N$.
\end{rmk}
\begin{dfn}$f: M\lrta N$ is a morphism of $\cala$-modules.\\
\begin{itemize}
\item 
$
Ker(f)=f^{-1}(\{0\})\subset M
$ 
is a submodule of $M$.
\item 
$
Im(f)= f(M)\subset N
$ is a submodule of $N$.
\item 
Coker(f)=N/Im(f) is an $\cala$-module.
\end{itemize}
\end{dfn}
\begin{rmk}\label{rmk:iso} 
\begin{enumerate}
\item $ker(f)={0}\Longleftrightarrow f$ is injective.
\item $coker(f)={0}\Longleftrightarrow f$ is surjective.
\item if $f:M\rta N$ and $M'\subset ker(f)$, then we get an induced linear map $\bar{f}$, s.t the diagram 
\[
\begin{tikzcd}
M \arrow[r,"f"] \arrow[d,"\pi"]  & N  \\
   M/M' \arrow[ur,"\bar{f}"] & 
\end{tikzcd}
\]
commutes. It properly defined by $\bar{f}(x+M')=f(x)$ since $f(M')=\{0\}$
Then we have
$$
Im(\bar{f})=Im(f),
$$
and 
$$
Ker(\bar{f})=Ker(f)/M'.
$$
In particular, if $M'= Ker(f)$, we get an isomorphism 
$$
M/Ker(f)\overset{\bar{f}}{\lrta}Im(f).
$$
\end{enumerate}
\end{rmk}

If $M$ is an $\cala$-module and $(M_i)_{i\in I} $ a family of submodules then $\cap_{i\in I} M_i$ is a submodule.
If $X\subset M$ be a  subset then the intersection of all submodules  containing $X$ is a submodule containing $X$, called the  submodule generated by $X$, denote it by $\lgl X\rgl$.
One checks that 
$$
\begin{aligned}
\lgl X\rgl&=\left\{\text{linear combination of elements of } X\right\}\\
&=\left\{\sum_i a_i x_i|K\geq 0 \intg, a_i\in\cala, x_i\in X\right\}
\end{aligned}
$$
We write 
$$
\sum_{i\in I} M_i=\lgl \cup_{i\in I} M_i\rgl
$$
\begin{dfn}
If $M$ satisfies $M=\lgl X\rgl$ with $X$ finite, then $M$ is called finitely generated.
\end{dfn}
Warning:   A submodule of a finitely generated module is not necessarily finitely generated.
\begin{ex}
$$
A=\cplx[X_1,...,X_n,...].
$$
$A$ is finitely generated by $1$ however, the ideal 
$
I=(X_1,..,X_n,...) $ is note finitely generated
\end{ex}

\begin{lemma}\ 
\begin{enumerate}
    \item $L\supset M\supset N$ are $\cala$-modules, then there is an isomorphism 
    $$
        (L/N)/(M/N)\rta L/M
    $$

    $$
    (x+N)+M/N\mapsto x+M
    $$
    Rigorously:
    $\pi: L\lrta L/M$ is surjective\\
    $\Lrta \bar{\pi} :L/N\rta L/M$ is surjective\\
    and $Ker(\bar{\pi})=M/N$ so
     $$
     (L/N)/(M/N)\cong Im(\bar{\pi}),
     $$
     by Remark~\ref{rmk:iso}.
    \item $(M_1+M_2)/M_2\cong M_1/(M_1\cap M_2)$
\end{enumerate}
\end{lemma}
\begin{dfn}
$I\subset A$ ideal; $M$ module $I M=\lgl \{ax|a\in I,x\in M\}\rgl\subset M$ as a submodule.
\end{dfn}
$M/IM$ is naturally an $\cala/I$-module.

\begin{dfn}
$(M_i)_{i \in I}$ is a family of $\cala$-modules
\begin{enumerate}
\item $\prod_{i\in I} M_i$ is an $\cala$-module with $a(x_i)=(a x_i)$.
\item $\oplus_{i\in I} M_i\subset \prod_{i\in I} M_i$ is the submodule of $(x_i)_{i\in I}$ s.t. $x_i=0$ for all but finitely many $i\in I$. 
\end{enumerate}
Cartesian product and direct product are the same when there only finitely many summand. If $M_i=M,\forall i\in I$,we denote $M^{(I)}:=\oplus_i M_i$. When $I$ is finite, we denote it by $M^I$.
\end{dfn}

\begin{dfn}
An $\cala$-module $M$ is called \textbf{free} if there exists a set $I$ s.t. $M$ is isomorphic to $\cala^{(I)}$.
\end{dfn}
\begin{ex}\ 
\begin{enumerate}
\item if $\cala$ is a field, then every $\cala$-module is free.
\item $\cala=\intg$ : $\intg/2\intg$ is not free.
\item \textbf{Warning!} A submodule of a free module is not necessarily free.(e.g. ideals in $\cala$)
\item If $\cala\neq \{0\}$, $n,m\geq 0$ are integer and $\cala^n\cong \cala^m$ then $n=m$.
$I\subset \cala$ maximal ideal, then we get an isomorphism of $\cala/I$-vector spaces,
$$
(\cala/I)^n\cong(\cala/I)^m\Lrta n=m.
$$
This is called the \textbf{invariant basis number property}, all nontrivial commutative ring has the property. 
\end{enumerate}
\end{ex}

\begin{prop}
(Nakayama's lemma)\\
$M$ finitely generated $\cala$-module, $I\subset \text{Jacobson radical of \cala, which is just } \cap_{\scm\subset\cala\ }\scm$, where $\scm $ are maximal ideals in $\cala$.
If $IM=M$, then $M=\{0\}$. e.g. 
$\cala$ being a local ring and $I=\scm$ the only maximal in $\cala$.
\end{prop}

\begin{proof}
Suppose $M\neq {0}$, and let $\{x_1,...,x_n\}$ be a generating set with $n\geq1$ minimal.
Since $IM=M$, we have $x_n\in IM$, so
$$
x_n=\sum_{i=1}^k
a_i y_i, y_i\in M, a_i\in I
$$
where $y_i=\sum_j b_{ij }x_j$.
Then we have 
$$
x_n=\sum_{j=1}^n c_j x_j
$$
$$
c_j=\sum_i^k a_i b_{ij}\in I
$$
$$
\Lrta (1-c_n)x_n=\sum_{j=1}^{n-1}c_j x_j
$$ 
and
$(1-c_n)\equiv 1\mod I$ $\Lrta$ $c_n\in$ the Jacobson radical, then  $1-c_n$ is invertible by Proposition~\ref{prop:criterion_jacobson_radical}.
$$
x_n=(1-c_n)^{-1}\sum_{j=1}^{n-1}c_j x_j,
$$ 
which contradict the minimality of the generating set.
\end{proof}

\begin{cor}
$M$ fin. gen. $\cala$-module, $I \subset \text{Jacobson radical }, N\subset M$. If $M=IM +N$, then $M=N$.
\end{cor}
\begin{proof}
$I(M/N)=(IM+N)/N=(M/N)$, then by Nakayama's lemma we know 
$$
M/N={0}.
$$
\end{proof}

\begin{cor}
$\cala$ local ring, $\scm\subset \cala$ the maximal ideal. $M$ fin. gen. Then if $(x_1,...,x_n)\in M$ are such that their classes modulo $\scm$ form a basis of $M/\scm M$ as $\cala/\scm$-vector space, then they generate $M$.
\end{cor}

\begin{proof}
$N=\lgl x_1,...,x_n\rgl$ and apply Nakayama's lemma.
\end{proof}
\section*{Exact sequence}
\begin{dfn}
(1)$M'\rta(f) M\rta(g) M''$ is \textbf{exact} if $Im(f)=ker(g)$\\
(2) $M'\rta(f_1) M\rta(f_2) M''\rta...$ is \textbf{exact} if it is exact at each node.
\end{dfn}

\begin{ex}\ 
\begin{enumerate}
\item $0\lrta M\overset{g}{\lrta} M''$ is exact, is equivalent to say that $g$ is injective
\item $M'\overset{f}{\lrta}M\lrta 0$ is exact, it is equivalent to say that $f$ is surjective.
\item ``Short exact sequence'' $0\lrta M'\overset{f}{\lrta}M\overset{g}{\lrta}M'' \lrta0$ For instance, 
$$
\begin{array}{ c c c c c c c c c}
0 & \longrightarrow  & M' & \overset{f}{\longrightarrow } & M'\oplus M'' & \overset{g}{\longrightarrow } & M'' & \longrightarrow  & 0\\
 &  & x &\longmapsto  & (x,0) &  &  &  & \\
 &  &  &  & (x,y) & \longmapsto & y &  &
\end{array}
$$
 the splitting sequence is exact. In fact short exact sequence of free modules always splits.
 \item $\cala=\intg$, for non-free modules, for example
 $$
\begin{array}{ c c c c c c c c c}
0 & \longrightarrow  & \intg/2\intg & \longrightarrow  & \intg/4\intg & \longrightarrow  & \intg/w\intg & \longrightarrow  & 0\\
 &  & x &\longmapsto  & 2x &  &  &  & \\
 &  &  &  & x & \longmapsto & x\  mod\ 2 &  &
\end{array}
$$
the exact sequence does not split.
\end{enumerate}
\end{ex}
\subsection{Lecture 4. Snake Lemma, Tensor Product by Professor Kowalski}
\begin{prop}(Snake Lemma)
Suppose we have such a commutative diagram, each row is exact, 
\[
\begin{tikzcd}
0\arrow[r]  & M'  \arrow[r]\arrow[d,"f'"] & M\arrow[r] \arrow[d,"f"] & M''\arrow[r] \arrow[d,"f''"] & 0 \\
0\arrow[r]  & N'  \arrow[r] & N\arrow[r] & N''\arrow[r]  & 0 
\end{tikzcd}
\]
then we have a map $\delta: Ker(f'')\lrta Coker(f')$ s.t.  
$$
0\lrta Ker(f')\lrta Ker(f)\rta Ker(f'')\overset{\delta}{\lrta}Coker(f')\lrta Coker(f)\lrta Coker(f'')\lrta 0 
$$
is exact.
\end{prop}

\begin{proof}
Consider the kernels and cokernels with the induced map between them. For notion consideration, we write $Ker(f')$ as $K'$ and $Coker(f')$ as $C'$ and so on. We have the extended commutative diagram:
\[
\begin{tikzcd}
0\arrow[r]  
& K'  \arrow[r,"\hat{u}"]\arrow[d,"k'",hook] 
& K\arrow[r,"\hat{v}"] \arrow[d,"k",hook] \arrow[ddd,phantom,""{coordinate, name=Z}] 
& K'' \arrow[d,"k''",hook]  
& \\
0\arrow[r]  & M'  \arrow[r,"u"]\arrow[d,"f'"] & M\arrow[r,"v"] \arrow[d,"f"] & M''\arrow[r] \arrow[d,"f''"] & 0 \\
0\arrow[r]  & N'  \arrow[r,"u'"] \arrow[d,"q'",two heads]& N\arrow[r,"v'"] \arrow[d,"q",two heads]& N''\arrow[r]\arrow[d,"q''",two heads]  & 0 \\
  & C'  \arrow[r,"\bar{u}"] & C\arrow[r,"\bar{v}"] & C''\arrow[r]  & 0,
\end{tikzcd}
\]
where the maps $k',k,k''$ are inclusion of the kernels as submodules and $q',q,q''$ are canonical projections, hence each column become exact now. $\bar{u}, \bar{v}$ are the morphism induced on quotient modules while $\hat{u}, \hat{v}$ are restrictions of $u,v$ on submodules. One can check the induced maps on Cokernels are well defined, for example, for $\bar{v}$ to be well defined, because $q''\circ v'\circ f=q''\circ f''\circ v =0$, thus $Im(f)\subset Ker(q''\circ v')$. One can also check that the above diagram is commutative. For example $x\in K'$, we have $f(\hat{u}(x))=f(u(x))=u'(f'(x))=0\Lrta \hat{u}(x)\in K$, then we have $u\circ k'=k\circ \hat{u}$. 
\begin{enumerate}
\item Exactness at $K'$\\
We already know $\hat{u}=u|_{Ker(f')}$, $u$ injective implies that $\hat{u}$ is injective.
\item Exactness at $K$\\
We easily check that $Im(\hat{u})\subset Ker(\hat{v})$, because $k''\circ \hat{v}\circ \hat{u}=v\circ u\circ k'=0$, by the fact $k''$ is injective, we know $\hat{v}\circ\hat{u}=0$. For the converse inclusion, if $x\in Ker(\hat{v})=Ker(v)\cap Ker(f)$, then $x\in Im(u)\cap Ker(f)$. $\exists y\in M'$ s.t. $u(y)=x$ $\Lrta f(u(y))=0 \Lrta u'(f'(y))=0 $. Then because $u'$ is injective, $f'(y)=0\Lrta y\in K'\Lrta x=\hat{u}(y)$. Then we conclude $Ker(\hat{v})\subset Im(\hat{u})$, thus $Ker(\hat{v})= Im(\hat{u})$.
\item Exactness at $C''$\\
$q''\circ v'=\bar{v}\circ q$, $q'', v', q$ are all surjective, then we conclude that $\bar{v}$ has to be surjective.
\item Exactness at $C$\\
We easily verify that $\bar{v}\circ \bar{u}=0$, i.e. $\bar{v}\circ \bar{u}\circ q'=q''\circ v'\circ u'=0$ and $q'$ is surjective $\Lrta \bar{v}\circ \bar{u}=0$. For the converse inclusion, we choose $x+Im(f)\in Ker(\bar{v})$, where $x\in N$. $\bar{v}(x+Im(f))=0= q''\circ v'(x)$. $v'(x)\in Ker(q'')=Im(f'')$. $\exists y\in M'' $ s.t. $f''(y)=v'(x)$, On the other hand, $v$ is surjective ,$\Lrta \exists z\in M$ s.t. $v(z)=y$. Then, we have $f''(v(z))=v'(x)=v'(f(z))$. Then we choose $\tilde{x}=x-f(z)$, $\Lrta x+Im(f)=\tilde{x}+Im(f)$ \& $v'(\tilde{x})=0$. Then there exists $w\in N'$ s.t. $u'(w)=\tilde{x}$. Then, we check that $q\circ u'(w)=q(\tilde{x})=\tilde{x}+Im(f)$, thus $\bar{u}(q(w))=\tilde{x}+Im(f)\Lrta \bar{u}(w+Im(f'))=x+Im(f)$. Then we conclude $Ker(\bar{v})\subset Im(\bar{u})$.
\item Construct $\delta$\\
\[
\begin{tikzcd}
0\arrow[r]  
& K'  \arrow[r,"\hat{u}"]\arrow[d,"k'",hook] 
& K\arrow[r,"\hat{v}"] \arrow[d,"k",hook] \arrow[ddd,phantom,""{coordinate, name=Z}] 
& K'' \arrow[d,"k''",hook]
\arrow[phantom]{d}[red, xshift=-1.5 ex]{\substack{x\\[0.5\jot]\downmapsto\\[0.5\jot]x}}
\arrow[dlldd,
"\delta",blue,
rounded corners,
to path=
{ -- ([xshift=12ex]\tikztostart.east)
|- (Z) \tikztonodes
-| ([xshift=-12ex]\tikztotarget.west)
-- (\tikztotarget)}
] 
& 
\\
0\arrow[r]  
& M'  \arrow[r,"u"]\arrow[d,"f'"] 
& M\arrow[r,"v"] \arrow[d,"f"] \arrow[phantom]{r}[red, yshift=-1.5 ex]{\substack{\exists y\ \mapsto\ x}} \arrow[phantom]{d}[red, xshift=-2 ex]{\substack{y\\[0.5\jot]\downmapsto\\[0.5\jot] f(y)}}
& M''\arrow[r] \arrow[d,"f''"] 
& 0 
\\
0\arrow[r]  
& N'  \arrow[r,"u'"] \arrow[d,"q'",two heads] \arrow[phantom]{r}[red, yshift=-1.5 ex]{\substack{\exists z\ \mapsto\ f(y)}} \arrow[phantom]{d}[red, xshift=-2 ex]{\substack{z\\[0.5\jot]\downmapsto\\[0.5\jot] [z]}}
& N\arrow[r,"v'"] \arrow[d,"q",two heads]
& N''\arrow[r]\arrow[d,"q''",two heads]  
& 0 
\\
& C'  \arrow[r,"\bar{u}"] 
& C\arrow[r,"\bar{v}"] 
& C''\arrow[r]  
& 0,
\end{tikzcd}
\]
For an element $x\in K''$, $k''(x)=x\in M''$ and $f''(x)=0$. $\because v$ is surjective, $\therefore \exists y\in M $ s.t. $v(y)=x$. Then $f''(x)=f''(v(y))=v'(f(y))=0$ $\Lrta f(y)\in Ker(v')=Im(u')$. Therefore, there exists $z\in N'$ s.t. $u'(z)=f(y)$. The choice of $z$ is unique once we fix $y$, because $u'$ is injective. \textbf{We define $\delta:K''\lrta C', x\mapsto [z]=z+Im(f')$}. For $\delta$ to be well defined, it can not depend on the choice of $y$ and $z$. Choose another $\tilde{y}\in M$ and corresponding $\tilde{z}\in N'$ s.t. $v(\tilde{y})=x$ and $u'(\tilde{z})=f(\tilde{y})$. We have $v(\tilde{y}-y)=0$, $\exists w\in M'$ s.t. $u(w)=\tilde{y}-y$. Then $f(u(w))=u'(f'(w))=f(\tilde{y}-y)=f(\tilde{y})-f(y)$. Then we have $u'(\tilde{z})-u'(z)=u'(f'(w))$. Since $u'$ is injective, we have $\tilde{z}=z+f'(w)$, thus $\tilde{z}+Im(f')=z+Im(f')$.  Then we conclude that $\delta$ is well defined.
\item Exactness at $K''$\\
For $x\in K$, we formally write
$$
\begin{aligned}
\delta(\hat{v}(x))&=u'^{-1}(f(v^{-1}(k''(\hat{v}(x)))))+Im(f')\\
&=u'^{-1}(f(v^{-1}(v(k(x)))))+Im(f')\\
&=u'^{-1}(f(k(x)))+Im(f')\\
&=0 \text{ because } f\circ k=0.\\
&\Lrta Im(\hat{v})\subset Ker(\delta)
\end{aligned}
$$
For the converse inclusion. $\forall x\in Ker(\delta)$, we trace back to the construction of $\delta$, and select the corresponding $y\in M,\ z\in N'$, where $v(y)=x$ and $u'(z)=f(y)$. $\because x\in Ker(\delta),\therefore z\in Im(f')$. $\Lrta \exists w\in M'$ s.t. $f'(w)=z$. Then we choose another $\tilde{y}=y-u(w)$, one verifies that $v(\tilde{y})=v(y)-v(u(w))=v(y)=x$. (this is legal, because we know $\delta$ does not depend on the choice of $y$) Also, we know $f(\tilde{y})=f(y)-f(u(w))=f(y)-u'(f'(w))=f(y)-u'(z)=0$. Then we know $\tilde{y}\in Ker(f)=K$, we conclude that $\hat{v}(\tilde{y})=x$, thus $Ker(\delta)\subset Im(\hat{v})$.
\item Exactness at $C'$\\
For $x\in K''$, we formally write
$$
\begin{aligned}
\bar{u}(\delta(x))&=\bar{u}\left(u'^{-1}(f(v^{-1}(k''(x))))+Im(f')\right)\\
&=(q\circ u')\left(u'^{-1}(f(v^{-1}(k''(x))))\right)\\
&=q(0+f(v^{-1}(k''(x))))\\
&=0\\
&\Lrta Im(\delta)\subset Ker(\bar{u})
\end{aligned}
$$
For the converse inclusion, we choose an element $z+Im(f')\in Ker(\bar{u})$. Then $\bar{u}(z+Im(f'))=q\circ u'(z)=0$, then we have $\exists y\in M \text{ s.t. } u'(z)=f(y)$. Also we have $v'(u'(z))=v'(f(y))=0,\Lrta f''(v(y))=0$. $v(y)\in Ker(f'')=K''$. We can check that $\delta(v(y))=z+Im(f')$. Hence, we conclude that $Ker(\bar{u})\subset Im(\delta)$.
\end{enumerate}
\end{proof}

\begin{ex}
(Application of snake lemma) We have such a commutative diagram, each row is exact. Suppose the middle map is isomorphism. 
\[
\begin{tikzcd}
0\arrow[r]  & M'  \arrow[r]\arrow[d,"f'"] & M\arrow[r] \arrow[d,"f"] & M''\arrow[r] \arrow[d,"f''"] & 0 \\
0\arrow[r]  & N'  \arrow[r] & N\arrow[r] & N''\arrow[r]  & 0 
\end{tikzcd}
\]
then we have a map $\delta: Ker(f'')\lrta Coker(f')$ s.t.  
$$
0\lrta Ker(f')\lrta \{0\} \rta Ker(f'')\overset{\delta}{\lrta}Coker(f')\lrta \{0\}\lrta Coker(f'')\lrta 0 
$$
is exact. Thus we get $\delta:Ker(f'')\lrta Coker(f')$ is an isomorphism.
\end{ex}
\begin{prop}

\end{prop}
If $0\lrta M'\overset{u}{\lrta}M\overset{v}{\lrta}M'' \lrta0$  is exact, then for any $\cala$-module $N$,
\begin{equation}\label{eq:1}\tag{*}
\begin{array}{ c c c c c c c c c}
0 & \longrightarrow  & Hom_\cala(M'',N) & \overset{v^*}{\longrightarrow } & Hom_\cala(M,N) & \overset{u^*}{\longrightarrow } & Hom_\cala(M',N)  \\
 &  & f &\longmapsto  & f\circ v &  &   \\
 &  &  &  & g & \longmapsto & g\circ u 
\end{array}
\end{equation}
is exact, in general $u^*$ is not surjective. Also,
\begin{equation}\label{eq:2}\tag{**}
\begin{array}{ c c c c c c c c c}
  Hom_\cala(N,M'') & \overset{u_*}{\longrightarrow } & Hom_\cala(N,M) & \overset{v_*}{\longrightarrow } & Hom_\cala(N,M') & \longrightarrow  & 0\\
  f &\longmapsto  & u\circ f  &  &  &  & \\
   &  & g & \longmapsto & v\circ g &  &
\end{array}
\end{equation}
is exact but $u_*$ is in general not always injective.
\\
More precisely, we have \textbf{right exactness of functor $Hom(\underline{\ \ }, N)$}:\\
$$
M'\overset{u}{\lrta}M\overset{v}{\lrta}M'' \lrta0 \text{ is exact} \Longleftrightarrow (\ref{eq:1}) \text{ is exact for all $N$}
$$\\
and 
\textbf{Left exactness of functor $Hom(N,\underline{\ \ })$}:\\
$$
0\lrta M'\overset{u}{\lrta}M\overset{v}{\lrta}M'' \text{ is exact }\Longleftrightarrow (\ref{eq:2})\text{ is exact for all $N$}.
$$
\begin{proof}
For ``$\Lrta$'' part of the first statement, we assume $M'\overset{u}{\lrta}M\overset{v}{\lrta}M'' \lrta0$ is exact. 
Let $N$ be $\cala$-module, then we check that:
\begin{enumerate}
\item $u^*\circ v^*=0$\\
Let $f:M''\lrta N$, $(u^*\circ v^*)(f)=f\circ v\circ u=f\circ(v\circ u)=0$ 
\item $v^*$ is injective\\
Let $f:M''\lrta N$ be such that $v^*(f)=f\circ v=0$ $\Lrta f(Im(v))=0$ $\Lrta$ $f=0$ because $v$ is surjective.
\item $Ker(u^*)\subset Im(v^*)$\\
Let $f:M\lrta N$ be such that $u^*(f)=f\circ u=0$. Then $f(Im(u))=0$ so $f(Ker(v))=0$, so there is $\bar{f}:M/Ker(v)\lrta N$ s.t. $\bar{f}\circ p=f$.
\[
\begin{tikzcd}
 M  \arrow[r,"f"] \arrow[d,"p"] & N \\
M/Ker(v)\arrow[swap, ur,"\bar{f}"] &    
\end{tikzcd}
\]
We know that $v$ induces an isomorphism 
\[
\begin{tikzcd}
 Im(v)=M''\arrow[dr,swap,bend right,"\bar{v}^{-1}"]&
 M \arrow[swap,l,"v"] \arrow[r,"f"]  \arrow[d,"p"] & N \\
& M/Ker(v)\arrow[swap, ur,"\bar{f}"] \arrow[ul,"\bar{v}"]& 
\end{tikzcd}
\]
Let $f'=\bar{f}\circ \bar{v}^{-1}\in Hom(M'',N)$, we compute $v^*(f')=f'\circ v=\bar{f}\circ \bar{v}^{-1}\circ v=\bar{f}\circ p=f$ thus $f\in Im(v^*)$
\end{enumerate} 
We then give an example where the surjectivity of $u^*$ fails\\
Consider $\cala=\intg$,\ 
$0\lrta \intg\overset{\times 2}{\lrta}\intg\overset{\ }{\lrta}\intg/2\intg \lrta0$ is exact.
$$
\begin{aligned}
v^*: & Hom(\intg, N)\rta Hom(\intg,N) \\
 &\ \ \ \ \ \ \ \ \ f \longmapsto  f\circ(\times 2)
\end{aligned}
$$
 is not surjective if $N=\intg$, because $f=Id_\intg$, we want to find a map $g$ such that the following diagram commutes,
 \[
\begin{tikzcd}
0 \ar[r] & \intg\ar[r,"\times 2"]\ar[d,"Id"] & \intg \ar[dl, dashed,"?"description, "g"] \\
& \intg &
\end{tikzcd}
 \]
 but there is no $g$ such that $g\circ(\times 2)=Id_\intg$ because every morphism in $Hom_\intg(\intg,\intg)$ is of the form $\times q$, where $q\in\intg$.

 Conversely, for the ``$\Llta$'' part of the first statement, assume (\ref{eq:1}) is always exact. We want to show that $M'\overset{u}{\lrta}M\overset{v}{\lrta}M'' \lrta0$ is exact, 
 $$
\begin{array}{ c c c c c c c c c}
0 & \longrightarrow  & Hom_\cala(M'',N) & \overset{v^*}{\longrightarrow } & Hom_\cala(M,N) & \overset{u^*}{\longrightarrow } & Hom_\cala(M',N) &  & \\
 &  & f &\longmapsto  & f\circ v &  &  &  & \\
 &  &  &  & g & \longmapsto & g\circ u &  &
\end{array}
 $$
 \begin{enumerate}
\item Let $N=Coker(v)$ and $[p:M''\lrta Coker(v)]\in Hom(M'',N)$, then $v^*(p=p\circ v=0)$. Since $v^*$ is injective, we have $p=0$, in other words $M''=Ker(p)=Im(v)$ so $v$ is surjective.
\item Take $N=M''$ and $f=Id_{M''}$, $(u^*\circ v^*)(f)=0$ means $Id_{M''}\circ v\circ u=0$ $\Lrta v\circ u=0$, hence $Im(u)\subset Ker(v)$.
\item  Take $N=M/Im(u)$, and $p:M\lrta N$ projection, we have $u^*(p)=p\circ u=0$. So $p\in Ker(u^*)$, so there exists $f\in Hom(M'',N) $ s.t. $v^*(f)=f\circ v=p$. 
\[
\begin{tikzcd}
 M'  \arrow[r,"f"] & N=M/Im(u) \\
M\arrow[ur,"p"] \arrow[u,"v"] &    
\end{tikzcd}
\]
Hence $Ker(v)\subset Ker(p)$ and $Ker(v)\subset Im(u)$, then we can conclude that $Ker(v)=Im(u)$.
\end{enumerate}
The above steps proves the first statement and proof of the second statement is similar.
\end{proof}
\section*{Tensor Product}
\begin{dfn}
$M,N,P$  are $\cala$-modules, A map $f:M\times N\lrta P$ is called $\cala$-\textbf{bilinear} if 
$$
f(ax+by, z)=af(x,z)+bf(y,z)
$$
$$
f(x,ay+bz)=a f(x,y)+bf(x,z)
$$
$$
Bil_{\cala}(M,N,P)=\{\text{ all $\cala$-bilinear maps form $M\times N$ to $P$}\}.
$$
$Bil_{\cala}(M,N,P)$ is an $\cala$-module.
\end{dfn}

\begin{dfn}\label{def:tensor_product}
$M,N$ are $\cala$-modules and the \textbf{tensor product} gives an $\cala$-module $M\otimes_\cala N$ such that $Bil_\cala(M,N;P)=Hom_{\cala}(M\otimes_\cala N,P)$. $Bil_\cala(M,N;P)$ is obviously an $\cala$-module, with sum and scalar multiplication performed valuewise.
\end{dfn}

\begin{thm}
$M,N$ are $\cala$-modules. There exists a pair $(T,\beta)$ where $T$ is an $\cala$-module and $\beta:M\times N\lrta T$ s.t.  any $\cala$-bilinear map $b:M\times N\lrta P$ factors through $(T,\beta)$, i.e. there exists a unique $f:T\lrta P$ s.t. the following  diagram commutes.
\[
\begin{tikzcd}
 M\times N \arrow[d,"\beta"] \ar[r,"b"] & P \\
T\arrow[swap,ur,"\exists !f"] &    
\end{tikzcd}
\]
This is what we call \textbf{universal property}. One can check that if it exists, it is unique.
\end{thm}
\subsection{Lecture 5. Properties of Tensor Product}
The motivation of tensor product is  to ``classify'' bilinear/multilinear maps between modules over some ring $\cala$.
\begin{dfn/thm}
$M$ and $N$ are $\cala$-modules, \textbf{there exists a best possible bilinear map $M\times N\rta M\otimes N$.} That is to say : there exists a module $T$ (denoted $M\otimes N$ or $M\otimes_\cala N$) and a bilinear map $f:M\times N\lrta T$. By ``best possible'', we mean:
For all module $P$ and all bilinear map $b:M\times N\rta P$,
here exists a unique $\tilde{b}:T\lrta P$ s.t. the following  diagram commutes.
\[
\begin{tikzcd}
 M\times N \arrow[d,"f"] \ar[r,"b"] & P \\
T\arrow[swap,ur,"\exists !\tilde{b}"] &    
\end{tikzcd}
\] 
What's more $(T,f)$ is \textbf{strongly unique} which means \textbf{it is unique up to unique isomorphism}
\[
\begin{tikzcd}
 M\times N \arrow[d,"f"] \ar[r,"f'"] & T'\arrow[swap,dl,"\exists !k"] \\
T\arrow[swap,ur,bend right,"\exists !j"] &    
\end{tikzcd}
\] 
\end{dfn/thm}
\begin{proof}
\textbf{Uniqueness}\\
The uniqueness is just the direct result of universal property. By definition, $f$ is bilinear. Apply the universal property with $P=T'$, $b=f'$, then we know $j:=\tilde{b}:T\rta T'$. Similarly, we can construct $k$ by swapping $T,T'$. Consider $k\circ j:T\rta T$, apply the universal property with $P:=T$, $b:=f$
\[
\begin{tikzcd}
 M\times N \arrow[d,"f"] \ar[r,"f"] & T \\
T\arrow[swap,ur,"\exists !\tilde{b}"] &    
\end{tikzcd}
\] 
We know $\exists ! \tilde{b}$ s.t. the diagram commutes. Then we have $\tilde{b}\circ f=f$, but another obvious map having this property is just $id_T$. Then, we get to the conclusion $k\circ j=id_T$ by the uniqueness of $\tilde{b}$. Similarly, we get $j\circ k=id_{T'}$. Altogether, we conclude that $(T,f)$ is unique up to unique isomorphism.\\
\textbf{Existence}\\
Form the free module $C:=\cala^{M\times N}$, where 
$$
\cala^{(M\times N)}=\left.\left\{\sum_{(x,y)\in M\times N} a_{(x,y)}(x,y)\right|a_{(x,y)}\in \cala, \text{almost all $a_{(x,y)}=0$}\right\}.
$$
{\color{red}We'd better mention the universal property of the free module $\cala^{(M\times N)}$, every map $q:M\times N\lrta P$ can be extended to $\tilde{q}:\cala^{(M\times N)}\lrta P$}\\
Let submodule $D\subseteq C$, then there is an induced map $\bar{g}:M\times N\lrta C/D$ for defining map $g:M\times N\lrta C$ of the free module. Then we consider a certain submodule $D$ with the following two equivalent definitions
\begin{itemize}
\item $D$ is the smallest submodule for which all the induced map $\bar{g}:M\times N\lrta C/D$ is bilinear.
\item $D$ it the submodule generated by the following elements
$$
\left\{\left.
\begin{aligned}
&(x+x',y)-(x,y)-(x',y)\\
&(x,y+y')-(x,y)-(x,y')\\
&a(x,y)-(ax,y)\\
&a(x,y)-(x,ay)
\end{aligned}
\right|\forall a\in \cala, \forall x,x'\in M,\forall y,y'\in N
\right\}
$$
\end{itemize}   
The equivalence of two definition can be explained by the definition of ``bilinear maps''. \\
We want to show that $C/D$ is what we are looking for. First, we claim, for all bilinear mop $ b:M\times N\rta P,\ Ker(\tilde{b})\supseteq D$.\\
The proof is to just check it by hand, e.g.
$$
\begin{aligned}
&\tilde{b}((x+x',y)-(x,y)-(x',y))\\
&=\tilde{b}((x+x',y))-\tilde{b}((x,y))-\tilde{b}((x',y))\\
&=b(x+x',y)-b(x,y)-b(x',y)\\
&=0(\text{by $b$ is bilinear})
\end{aligned}
$$
The characterization of $\tilde{b}$ determines its restriction of $g(M\times N)\subseteq T$. Clear by construction that $g(M\times N)$ generates $T$. We get the conclusion that $\bar{g}:M\times N\rta C/D=T$.
\end{proof}
Also note that, in general
$$
S:=\{m\otimes n|(m,n)\in M\times N\}\neq M\otimes N
$$, e.g. $\intg^n\otimes \intg/2\intg\cong \intg/2\intg$
but $S$ generates $M\otimes N$ as we saw in the proof.
\begin{ex}
Natural isomorphisms, $\exists !\ isomorphisms$
\begin{enumerate}
\item $M\otimes N\cong N\otimes M$
\item $(M\otimes N)\otimes P\cong M\otimes (N\otimes P)$
\item $M\otimes (N_1\oplus N_2)\cong (M\otimes N_1)\oplus (M\otimes N_2)$
\item $\cala\otimes M\cong M$
\end{enumerate}
\end{ex}
\begin{proof}
we prove part 3. Consider a map: 
$$
\begin{aligned}
b:M\times (N_1\oplus N_2)&\rta M\otimes N_1 \oplus M\otimes N_2\\
(m,(n_1,n_2))&\mapsto (m\otimes n_1, m\otimes n_2).
\end{aligned}
$$
We can check that $b$ is bilinear, for example 
$$
\begin{aligned}
&b(m+m',(n_1,n_2))\\
&=((m+m')\otimes n_1,(m+m')\otimes n_2)\\
&= (m\otimes n_1+m'\otimes n_1,m\otimes n_2+m'\otimes n_2)\\
&=(m\otimes n_1,m\otimes n_2)+(m'\otimes n_1,m'\otimes n_2)\\
&=b(m,(n_1,n_2))+b(m',(n_1,n_2)).
\end{aligned}
$$
As a result the bilinear map $b$ must factor through $M\otimes (N_1\oplus N_2)$, and we denote the corresponding map $f:M\otimes (N_1\oplus N_2)\rta M\otimes N_1 \oplus M\otimes N_2$.
$$
f(m\otimes (n_1,n_2))=(m\otimes n_1,m\otimes n_2).
$$
We use the terminology \textbf{pure tensor} to name the tensors like $x\otimes y\in M\otimes N$, obviously, $M\otimes N$ is linearly generated by  pure tensors.
We want to show that $f$ is an isomorphism. Need to find the inverse map $g$ of $f$.

define 
$$
\begin{aligned}
g_1:M\otimes N_1&\lrta M\otimes(N_1\oplus N_2)\\
(m\otimes n_1)&\longmapsto m\otimes (n_1,0)
\end{aligned}
$$
similarly, we can construct 
$$
\begin{aligned}
g_2:M\otimes N_2&\lrta M\otimes(N_1\oplus N_2)\\
(m\otimes n_2)&\longmapsto m\otimes (0,n_2)
\end{aligned}
$$
Then, we define $g=g_1\oplus g_2$. We want to show $f\circ g=id, g\circ f=id$.
$$
\begin{aligned}
&f\circ g(m\otimes n,m'\otimes n_2)\\
&=f(m\otimes (n_1,0)+m'\otimes(0,n_2))\\
&=(m\otimes n_1,0)+(0,m'\otimes n_2)\\
&=(m\otimes n_1,m'\otimes n_2)
\end{aligned}
$$
Then $f\circ g=id $ on pure tensors, hence it is identity on all tensors, because $f\circ g$ is linear, and pure tensor generates the whole tensor product module.
\end{proof}
Consider $\cala^m=\cala\oplus \cala\oplus...\oplus \cala$ (finite free module), by the isomorphism 4 in the above example 
$$
\begin{aligned}
\cala\otimes \cala&\cong \cala\\
x\otimes y&\mapsto xy
\end{aligned}
$$
also by iterating (3) and (4), we get 
$$
\cala^m\otimes \cala^n\cong \cala^{mn},
$$
compared to the known result
$$
\cala^m\oplus \cala^n\cong \cala^{m+n}.
$$
More directly,
if $e_1^{(1)},...,e_m^{(1)}$ standard basis for $\cala^m$, $e_1^{(2)},...,e_n^{(2)}$ standard basis for $\cala^n$,
then
$$
\left.\left\{e_i^{(1)}\otimes e_j^{(2)}\right|,m\geq i\geq1,n\geq j\geq 1\right\}
$$
form a basis of $\cala^m\otimes \cala^n$ and induces $\cong \cala^{mn}$

To see this directly, consider a bilinear map $f:\cala^m\times \cala^n\lrta P$, where $P$ is some module.
$$
\cala^m\ni x=x_1 e_1^{(1)}+...+x_m e^{(1)}_m, \ x_i\in \cala
$$
$$
\cala^n\ni y=y_1 e_1^{(1)}+...+y_m e^{(1)}_m, \ y_i\in \cala
$$
Then 
$$
f(x,y)=\sum_{
\begin{aligned}
&i=1...m\\
&j=1...n
\end{aligned}}
x_i y_j f(e_i^{(1)}\otimes e_j^{(2)}),
$$
where we can define $f(e_i^{(1)}\otimes e_j^{(2)})=:a_{ij}\in P$
Generally, given an $mn$-tuple $(a_{ij})$ in $P$ we may define a bilinear $f:\cala^m\times\cala^n\lrta P$ by the above formula.
\[
\begin{tikzcd}
 \cala^m\times \cala^n \arrow[d,"f"] \ar[r]
 \arrow[phantom]{r}[ yshift=3.5 ex]{\substack{ (e_i^{(1)}, e_j^{(2)})\ \longmapsto\ e_i^{(1)}\otimes e_j^{(2)}}} 
 & \cala^{\oplus\{e_i^{(1)}\otimes e_j^{(2)}\}} \arrow[dl,"\exists !\tilde{f}\ s.t.\ \tilde{f}(e_{ij})=a_{ij}"]
 \\
P &    
\end{tikzcd}
\] 
\begin{rmk}
More generally, we may define the $n$-fold tensor products $M_1\otimes ...\otimes M_n$.
$$
\left\{\text{multilinear maps :}M_1\times...\times M_n\lrta P \right\}\leftrightarrow \left\{\text{linear maps :}M_1\otimes...\otimes M_n\lrta P \right\}
$$
Let $V=\reals^n$, then 
$$
\left\{\text{inner products on }V \right\}\leftrightarrow \left\{\text{linear functions on }V\otimes V\right\}
$$
\end{rmk}
\begin{rmk}
\textbf{Extension of scalars}
Consider a ring morphism $f:\cala\rta\mathcal{B}$ and an $\cala$-module $M$, we can construct a $\calb$-module
$$
M_\calb:=M\otimes_\cala \calb,
$$
where $\calb$ is regarded as an $\cala$-module via $f$, i.e. $a\cdot b=f(a)b$. And the $\calb$ action on $M_\calb$ is like
$b\cdot(m\otimes z):=m\otimes b z $
\end{rmk}
\begin{ex}\ 
\begin{itemize}
\item $M=\cala^m\Lrta M_\calb=\calb^m$
\item $\cala=\reals, \calb=\cplx\Lrta (\reals^n)_\cplx:=(\reals^n)\otimes_\reals \cplx\cong\cplx^n$
\end{itemize}
\end{ex}

\subsection{Lecture 6. Flatness}
The meaning of $x\otimes y$ depends on the modules to which we regard $x$ and $y$ are belonging. In fact, one can have $x\in M'\subseteq M$ and $y\in N'\subset N$ but 
$$
M'\otimes N'\ni x\otimes y\neq x\otimes y\in M\otimes N
$$
\begin{ex}
$\cala=\intg$ , $M'=2\intg\subseteq M=\intg$, $N'=\intg/2 =N$,
 then $2\intg \otimes \intg/2\intg \ni 2\otimes 1\neq 0 $, but $\intg \otimes \intg/w\intg\ni2\otimes 1=0$ 
\end{ex}
In summary, we no $M'\subset M, N'\subset N$ does not indicate that $M'\otimes N'\subset M\otimes N$, which means the simple inclusion is  not an injective  morphism. 

But $\otimes$ is indeed a \textbf{bifunctor}. Given module morphisms
$$
\begin{aligned}
& f: M'\lrta M\\
& g: N'\lrta N\\
\exists ! f\otimes g:&M'\otimes N':\lrta M\otimes N\\
&x\otimes y\longmapsto f(x)\otimes g(y)
\end{aligned}
$$ 
and 
$$
(f\circ f')\otimes(g\circ g')=(f\otimes g)\circ (f'\otimes g')
$$
For example, we alway consider the case $g=1_N$ with $N$ $\cala$-module, then each morphism $f:M'\lrta M$ is mapped to $f\otimes 1_N:M'\otimes N\lrta M\otimes N$.

\begin{dfn}
$N$ is \textbf{flat} if $\forall f: M'\lrta M$  s.t.
$$
f:\text{injective }\Lrta f\otimes 1_N \text{ is injective}
$$
In other words,
 $$
M'\subset M\Lrta \text{``$M'\otimes N\subset M\otimes N$''}
 $$
\end{dfn}

\begin{ex}\ 
\begin{itemize}
    \item 
$\{0\}$ is a flat $\cala$-module
\item 
$\cala$ is a flat $\cala$-module, because $M\otimes_\cala\cala=M$ and $f=f\otimes 1_\cala$ 
\end{itemize}
\end{ex}

\begin{lemma}
Let $(N_i)_{i\in I}$ be a family of modules over $\cala$, then $\oplus_{i\in I} N_i$ is flat iff each $N_i$ is flat.
\begin{proof}
Suppose each $N_i$ is flat. Let $M'\overset{f}{\lrta}M$ be injective. Suppose, 
$$
M'\otimes(\oplus_{i}N_i)\overset{f\otimes 1}{\lrta}M\otimes (\oplus_i N_i) 
$$
is not injective, i.e.  $z\in Ker(f\otimes 1_N)\neq 0$. 
Let $N$ denote $\oplus_i N_i$ and the $i$-th projection $\pi_i:N\lrta N_i$.
\[
\begin{tikzcd}
0\neq z\arrow[r,phantom,"\in"] & \oplus_i (M'\otimes N_i)\arrow[r,"\rho_i'"] \arrow[d,equal] & M'\otimes N_i\arrow[d,equal] \\
& M'\otimes(\oplus_i N_i)\arrow[r,"1_{M'}\otimes \pi_i"] \arrow[d,"f\otimes 1_N"] & M'\otimes N_i \arrow[d,"f\otimes 1_{N_i}"] \\
& M\otimes (\oplus_i N_i)\arrow[r,"1_M\otimes \pi_i"] & M\otimes  N_i\\
& \oplus_i (M\otimes N_i)\arrow[r,"\rho_i"] \arrow[u,equal] & M\otimes N_i\arrow[u,equal] \\
\end{tikzcd}
\]
$z\neq 0\Lrta \exists i\in I$ s.t. $\rho_i'(z)\neq0$ $\Lrta (f\otimes 1_{N_i})(\rho_i'(z))\neq0\in M\otimes N_i$.
But $(f\otimes 1_{N_i})(\rho_i'(z))=\rho_i(f\otimes 1_{N}(z))$ is the $i$-th component of $(f\otimes1_N)(z)=0$ by assumption, which gives the contradiction.
The converse is simpler.
\end{proof} 
\end{lemma}
\begin{cor}
If $M$ is a free $\cala$-module, then it is a flat module.
\begin{proof}
We already know $\cala$ is flat, then by the previous lemma, we know $\oplus_{i\in I}\cala$ is flat.
\end{proof}
\end{cor}

\begin{ex}
Consider a system of linear equations
$$
S:f_1(x_1,...,x_n)=...=f_m(x_1,...,x_n)=0,
$$
where these $f_i$'s has coefficients in $\reals$. Then $S$ has solution over $\reals$ iff $S$ has solution over $\cplx$ (This claim works for any field extension $L/K$ instead of  $\cplx/\reals$)
A simple proof goes like: ``$\Lrta$'' is trivial, for the converse, we take the real or the imaginary part of a complex solution. 

For a second proof:
$$
M'=\reals^n\overset{f}{\lrta} M=\reals^m,
$$
where $f=(f_1,...,f_m)$.
$\cala=\reals$, $ N=\cplx\cong \reals\oplus \reals i$ is free, then by the above corollary, we know $N$ is flat.
Then $S$ has a solution over $\reals$ iff $Ker(f)\neq 0$, and $S$ has a solution over $\cplx$ iff $Ker(f\otimes 1_\cplx)\neq 0$. If $f\otimes 1$ is not injective, by the definition of flat module, we know $f$ is not injective, which conclude the proof. This second proof works for arbitrary field extension, because the field extensions are always free modules over the initial field.
\end{ex}

\begin{prop}(Right exactness of $\otimes N$)\\
Consider an exact sequence of $\cala$-modules
\begin{equation*}
\begin{array}{ c c c c c c c c c}
  M' & \overset{f}{\longrightarrow } & M & \overset{g}{\longrightarrow } & M'' & \longrightarrow  & 0\\
\end{array}
\end{equation*}
Then we have
\begin{equation*}
\begin{array}{ c c c c c c c c c}
  M'\otimes N & \overset{f\otimes 1}{\longrightarrow } & M\otimes N & \overset{g\otimes 1}{\longrightarrow } & M''\otimes N & \longrightarrow  & 0\\
\end{array}
\end{equation*}
 is  exact for arbitrary $\cala$-module $N$.
\end{prop}
\begin{proof}
Obviously $g\otimes 1$ is surjective. We only need to prove the exactness at $M\otimes N$. As for the easier inclusion,
$
Im(f\otimes 1)\subseteq Ker(g\otimes 1)
$
because $(g\otimes 1)\circ (f\otimes 1)=(g\circ f)\otimes 1=0$. 
Then it remains to show 
$$
\frac{M\otimes N}{Im(f\otimes 1)}\overset{\psi}{\lrta}M''\otimes N
$$
is an isomorphism. $\psi $ is induced by $g\otimes 1$,  well defined because $
Im(f\otimes 1)\subseteq Ker(g\otimes 1).
$

Now, we construct a two-sided inverse $\varphi$ of $\psi$.
\[
\begin{tikzcd}
 M''\otimes N  \ar[r,dashed,"\exists \varphi"] & \frac{M\otimes N}{Im(f\otimes 1)} \\
 M''\times N \arrow[u] \arrow[ur,dashed,"\exists \varphi_0"]& \\
M\times N\arrow[swap,uur,"\varphi_1"]\arrow[u,"g\times 1"] &    
\end{tikzcd}
\] 
Consider the map $\varphi_1$, it is the composition of the canonical projection and the defining map of tensor product.
$\varphi_1(x,y)\mapsto x\otimes y+Im(f\otimes 1)$. Consider $(x'',y)\in M''\times N$, which is the image of $(x,y)$ under $g\times 1$. Then we can define $\varphi_0(x'',y):=\varphi_1(x,y)$. It is well-defined, because if there is another $(x_1,y)$ also map to $(x'',y)$, the difference
$$
x-x_1\in Ker(g)=Im(f),
$$
hence $\exists z\in M'$
$x-x_1=f(z)$.
$\Lrta (x-x_1)\otimes y=(f\otimes 1)(z\otimes y)$
Then
$$
\varphi_1(x,y)-\varphi(x_1,y)=(x-x_1)\otimes y+Im(f\otimes 1)=0.
$$
Then it remains to check $\varphi_0$ is bilinear so that $\varphi_0$ lifts to a $\varphi$ on $M''\otimes N$. Also we  need to check the $\varphi$ is indeed the two-sided inverse of $\psi$.

Consider  $\varphi_0(x'',ay+bv)$ and $\varphi_0(ax''+bw'',y)$. Chose $x$ and $w$ in the preimages $g^{-1}(x'')$ and $g^{-1}(w'')$. By the linearity of $g$, we can safely choose $ax+bw$ in the pre-image of $ax''+bw''$
Knowing that $\varphi_1$ is bilinear (because the defining map of tensor product is bilinear and canonical projection is linear), we have
$$
\begin{aligned}
&\varphi_0(x'',ay+bv)=\varphi_1(x,ay+bv)\\
&=a\varphi_1(x,y)+b\varphi_1(x,v)=a\varphi_0(x'',y)+b\varphi_0(x'',v)
\end{aligned}
$$
and
$$
\begin{aligned}
&\varphi_0(ax''+bw'',y)=\varphi_1(ax+bw,y)\\
&=a\varphi_1(x,y)+b\varphi(w,y)=a\varphi_0(x'',y)+b\varphi_0(w'',y).
\end{aligned}
$$ 
Explicitly, with $x\in g^{-1}(x'')$, 
$$
\varphi(x''\otimes y)=x\otimes y+Im(f\otimes 1)
$$
and
$$
\psi(x\otimes y+Im(f\otimes 1))=g(x)\otimes y
$$
$$
\begin{aligned}
&\Lrta\\
&\psi\circ\varphi(x''\otimes y)=g(x)\otimes y=x''\otimes y\\
&\varphi\circ \psi(x\otimes y+Im(f\otimes 1))=x_1\otimes y+Im(f\otimes 1)=x\otimes y+Im(f\otimes 1),
\end{aligned}
$$
where in the last line $x_1$ is another representative in $g^{-1}(x'')$.
\end{proof}

\begin{cor}
$N$ is flat iff $\otimes N$ preserves the exactness of any sequence of modules
\begin{proof}
Any exact sequence can be split up into short exact sequence, and the flatness does indicate it preserve  the exactness of short exact sequence.
\end{proof}
\end{cor}

\begin{ex}\label{ex:iso_quotient_module}
An ideal
$\sca\subset \cala$, and $M$ is an $\cala$-module,
$$
M\otimes_\cala \cala/\sca\cong M/\sca M,
$$
where $\sca M:=\{\sum x_i m_i|x_i\in \sca, m_i\in M\}$. $\sca M$ is a submodule of $M$.
\begin{proof}
$$
0\lrta \sca\lrta \cala\lrta \cala/\sca\lrta 0 
$$
is an exact sequence (of \cala-modules). Tensorring it with $M$, we have
$$
\sca\otimes M\overset{\psi}{\lrta }M\lrta M\otimes \cala/\sca\lrta 0
$$
is exact, where $\psi $ is induced by the inclusion $\sca\inj\cala$, $\psi:x\otimes m\mapsto xm$. $Im(\psi)=\sca M$
Then by the exactness, we have
$$
M\otimes \cala/\sca\cong M/Im(\psi)=M/\sca M.
$$
\end{proof}
\end{ex}


\begin{ex}
$$\intg/m\intg\otimes_\intg \intg/n\intg\cong \intg/gcd(m,n)\intg.$$ 
Pf. Take $M=\intg/m\intg$, $\cala=\intg$, $\sca=n\intg$. Then $\sca M=(n\intg+m\intg)/m\intg=gcd(m,n)\intg/m\intg$. $\cala/\sca=\intg/n\intg$

Then by the result of Example~\ref{ex:iso_quotient_module}, we have
$$
M\otimes \cala/\sca=\frac{\intg}{m\intg}\otimes_\intg \frac{\intg}{n\intg}\cong\frac{\intg/m\intg}{gcd(m,n)\intg)/m\intg}= \frac{\intg}{gcd(m,n)\intg}=M/\sca M.
$$
\end{ex}

Let $n\in\intg$. Then $\intg/n\intg$ is flat iff $n=\pm1,0$, i.e. $\intg/n\intg=\{0\} \text{ or } \intg$. This is easy to prove, consider the following short exact sequence for $|n|\geq 2$,
$$
0\lrta n\intg\lrta \intg\lrta \intg/n\intg\lrta 0,
$$
Suppose $\intg/n\intg$ is flat. Tensorring it with the above exact sequence, we get
$$
0\lrta0\lrta \intg\otimes\intg/n\intg\lrta \intg/n\intg\lrta 0,
$$
which gives the contradiction. 

Fact\\
Any finitely generated $\intg$-module is of the form
$$
M=\intg^r\left(\oplus_{i\in I}(\intg/n_i\intg\right))
$$,
 the second part of $M$ is denoted $M_{tors}$, then we get the corollary that a finitely generated $\intg$-module is flat iff $M_{tors}$ vanishes.
\begin{dfn}
$\cala$ a ring, $M$ an $\cala$-module, we call $M$ \textbf{torsion free} if $\forall a\in \cala$ non-zerodivisor. $m\in M$ $am=0\Lrta m=0$
\end{dfn}

\begin{thm}\ 
\begin{enumerate}
\item $M$ if flat $\Lrta$ $M$ is torsion free
\item If $\cala$ is PID, $M$ is torsion free $\Lrta $ $M$ is flat.
\end{enumerate}
\end{thm}
\begin{proof}
Bosch section 4.2
\end{proof}

Some other facts about tensor product
\begin{ex}
For $\cala=\bbf$ being a field, $V, W$ finite dimensional vector space over $\bbf$
$$
\begin{aligned}
V^*\otimes W&\cong Hom_{\bbf}(V,W)\\
l\otimes w &\mapsto [v\mapsto l(v)w]
\end{aligned}
$$
\end{ex}


\section{Localization}

\subsection{Lecture 7 : Localization of rings}
\textbf{Motivation}
For $\cala$ an integral domain, we defined the quotients field $Frac(\cala)$. In general, one may want to invert part of $\cala$. For example, we may consider $\intg[1/2]=\{a/(2^n)|a\in \intg and n\in \mathcal{N}\}$. Each $2^n\in \intg[1/2]$ is invertible. For a subset $0\notin S\subseteq \cala$, we can define $\cala[1/S]$ to be the subring of $Frac(\cala)$ generated by $\cala$ and $\{1/s|s\in S\}$.

\begin{dfn}
A set of $\cala$, $S$ is \textbf{multiplicatively closed} if 
\begin{itemize}
\item $1\in S$
\item $s,t\in S\Lrta st\in S$
\end{itemize}
For a set $S\subset \cala$, we can define its \textbf{multiplicative closure}
$$
\overline{S}:=\{s_{I}|I=(i_1,...,i_n), \forall{n},  s_{i_n}\in S\}
$$
A set $S$ is multiplicatively closed iff $S=\overline{S}$. And we see that $\cala[1/S]=\cala[1/\overline{S}]$.
\end{dfn}


\begin{dfn}
Let $\cala$ be a ring $S\subseteq \cala$ a multiplicatively closed set, define a relation $\sim $ on $\cala\times S$:
$$
(a,s)\sim(a',s')\Longleftrightarrow\exists t\in S \ \text{s.t. } as' t= a' st
$$
\end{dfn}

\begin{lemma}
``$\sim$'' is indeed a equivalence relation.
\end{lemma}
\begin{proof}
reflectivity and symmetricity are trivial, for the transtivity
$$
(a,s)\sim(a',s')\sim (a'',s'')
$$
$$
\begin{aligned}
&\Lrta\\
&\exists t\in S: as't=a's t\\
& \exists t'\in S: a's'' t'=a''s't'\\
&as''(tt's')=as'ts''t'=a's''t's t=a''s (t t's')\\
& \Lrta (a,s)\sim(a'',s'')
\end{aligned}
$$
\end{proof}
\begin{dfn}
We define 
$S^{-1}\cala:(\cala\times S/\sim)$.  And we denote the equivalence class of $(a,s)$ by $a/s$.
\end{dfn}
\begin{prop}
There are well defined maps:
$$
+:S^{-1}\cala\times S^{-1}\cala\lrta S^{-1}\cala,\ (a/s,a'/s')\mapsto \frac{as'+a's}{ss'}
$$
$$
\cdot:S^{-1}\cala\times S^{-1}\cala\lrta S^{-1}\cala,\ (\frac{a}{s},\frac{a'}{s'})\mapsto \frac{aa'}{ss'}
$$
$$
0_{S^{-1}\cala}=\frac{0}{1}\text{ and } 1_{S^{-1}\cala}=\frac{1}{1}
$$
Then $(S^{-1}\cala,0_{S^{-1}\cala},1_{S^{-1}\cala},+,\cdot)$ is a ring.
\end{prop}
One can check that the above ring operation and $0,1$ are well-defined. e.g.\\
$$
\begin{aligned}
&\frac{a}{b}\cdot\frac{0}{1}\overset{?}{=}\frac{0}{1}\\
\Llrta & \frac{a\cdot 0}{b\cdot 1}\overset{?}{=}\frac{0}{1}\\
\Llrta & \frac{0}{b}\overset{?}{=}\frac{0}{1}\\
\Llrta &\exists t\in S: 0\cdot 1\cdot t=0\cdot b\cdot t
\end{aligned}
$$


We say $S^{-1}\cala$ is \textbf{localization of $\cala$ with respect to $S$}.
When $\cala$ is an integral domain, $S=\cala-\{0\}$ is multiplicative closed, the $S^{-1}\cala=Frac(\cala)$.
\begin{lemma}
There exists a ring morphism $\iota$ from $\cala$ to $S^{-1}\cala$ s.t each $a\in \cala$ maps to $a/1\in S^{-1}\cala$.
It has to following property
\begin{enumerate}[label=(\alph*)]
\item $\iota(S)\subset (S^{-1}\cala)^{\times}$
\item $Ker(\iota)=\{a\in \cala|sa=0 \text{ for some }s\in S\}$
\item Suppose $\cala\neq\{0\}.$ Then $\iota$ is injective $\Longleftrightarrow$ $S$ contains no zero divisors.
\item $S^{-1}\cala=\{0\}$ $\Longleftrightarrow$ $S\ni 0$
\item $\iota$ is isomorphism  $\Longleftrightarrow $ $S\subseteq \cala^{\times}$
\end{enumerate}
\end{lemma}
\begin{proof}We can easily check that $\iota$ thus defined is indeed a ring morphism.
\begin{enumerate}[label=(\alph*)]
\item  $s\in S$. $\iota(s)=s/1$ and $s/1\cdot 1/s=1$, then $s$ is a unit.
\item $a\in Ker(\iota)=\{b\in \cala|\frac{b}{1}=\frac{0}{1}\}$ $\Longleftrightarrow$ $\exists t\in S: t(a1-01)=ta=0\Llrta a\in\{\text{Zero divisors in }\cala\}$.
\item  derived from (a) and (b).
\item $S^{-1}\cala=\{0\}$ $\Longleftrightarrow$ $\frac{0}{1}=\frac{1}{1}$ $\Llrta$  there exists an element $t\in S$ s.t. $t\cdot1=0$, $\Longleftrightarrow S\ni 0$.
\item ``$\Lrta$'' Suppose $\cala\neq \{0\}$, then $\iota $ is isomorphism $\Llrta$  $\iota$ is surjective and injective $\Llrta$ $\forall \frac{a}{s}\in S^{-1}\cala: \exists c\in \cala\ s.t.\ \frac{a}{s}=\frac{c}{1}$ and $S$ is has no zero divisors. Then we know, $\frac{1}{s}=\frac{c}{1}\Lrta\exists  t(s\cdot c-1)=0$, and by the fact $S$ has no zero divisors $s\cdot c=1$, which means $S\subseteq \cala^{\times}$.\\
``$\Llta$'' Assume $\cala\neq\{0\}$. $S\subseteq\cala^\times$, then $S$ does not contain any zero divisors. $\forall \frac{a}{s}\in S^{-1}\cala, \exists v\in S \text{ s.t. } sv=1$. Then $ \frac{a}{s}=\frac{av}{1}\in Im(\iota)$, because $a s v=a$.



\end{enumerate}
\end{proof}

\begin{ex}
$X$ any set $U\subseteq X$ any subset.
$\cala:=\{\text{functions } f:X\lrta \reals\}$ is a ring of the the multiplication is defined value-wisely, $S:=\{f\in \cala| f(x)\neq 0, \forall x\in U\}$ is multiplicatively closed. Question, what is the localization $S^{-1}\cala$?
\end{ex}
\begin{lemma}
Let $B:=\{\text{functions }U\lrta \reals\}$.
Then the natural map $j:S^{-1}\cala\lrta B$ is an isomorphism $\frac{a}{s}\mapsto [U\ni x\mapsto\frac{a(x)}{s(x)}\in\reals]$
\end{lemma}
\begin{proof}
$j$ is well-defined:
Say $\frac{a}{s}=\frac{a'}{s'}$. Thus $\exists t\in S, a s' t=a' s t/$. Then $(a(x)s(x)-a'(x)s'(x))t(x)=0$, where $t(x)\neq0\forall x\in U$. Then by the properties of real numbers $\frac{a(x)}{s(x)}=\frac{a'(z)}{s'(x)}$.

Try defining $k:B\lrta S^{-1}\cala,\ b\longmapsto \tilde{b}/1$, where
$$
\tilde{b}:X\lrta \reals
$$
$$
\tilde{b}=\left\{
\begin{aligned}
&b(x),\ x\in U\\
&0, \ x\notin U
\end{aligned}
\right.
$$

$j\circ k=1$, $b\in B$
$\frac{\tilde{b}(x)}{1(x)}=b(x)\forall x\in U$

$k\circ j=1$ Say $b=j(\frac{a}{s})$, what we want is $\tilde{b}/1=a/s$, i.e. $\exists t\in S:(a\cdot 1-\tilde{b}\cdot s)t=0$.

Take $t:1_U=[x\mapsto 1 \text{ for }1\in U \text{ and } 0 \text{ for } x\not \in U]$
\end{proof}

\subsection*{Universal property of localization}
Recall $Hom(M\otimes N, P)\cong \{bilinear M\times N\lrta P\}$ and $Hom(\oplus_{i}M_i,N)\cong \prod_i Hom(M_i,N)$.

\begin{lemma}
$Hom(S^{-1}\cala,\calb)\cong \{f:\cala\lrta \calb\text{ s.t. } f(S)\subseteq \calb^{\times}\}$ 
where an element $\tilde{f}\in Hom(S^{-1}\cala,\calb)$
$$
\tilde{f}\left(\frac{a}{s}\right):=f(a)f(s)^{-1}
$$
$$
f(a):=\tilde{f}\left(\frac{a}{1}\right).
$$
i.e. every morphism $f:\cala\lrta\calb$ s.t. $f(S)\subseteq \calb^{\times}$, there exists a unique morphism $\tilde{f}:S^{-1}\cala\lrta \calb$ s.t. $f=\tilde{f}\circ\iota$, where $\iota $ is the canonical morphism $\iota:\cala\lrta S^{-1}\cala:a\mapsto \frac{a}{1}$.
\[
\begin{tikzcd}
S\ar[r,hook] &\cala  \arrow[r,"f"] \arrow[d,"\iota"] & \calb \\
&T\arrow[ur,swap,"\exists !\tilde{f}"]  &    
\end{tikzcd}
\]
 
This universal property of localization can serve as an alternative definition of localization. $S^{-1}\cala$ is defined to be a pair $(T,\iota)$

\end{lemma}
\begin{proof}
Want: $\forall f$ as above $\exists ! \tilde{f}\text{ s.t. } \tilde{f}\circ\iota=f$

Uniqueness:\\
$\tilde{f}(a/s)=\tilde{f}(a/1)\tilde{f}(s/1)^{-1}=f(a)f(s)^{-1}$

Existence :\\
Take $\tilde{f}(a/s):=f(a)f(s)^{-1}$, check that it is well defined:
$$
\frac{a}{s}=\frac{a'}{s'}\overset{?}{\Lrta} f(a)f(s)^{-1}=f(a')f(s')^{-1}
$$
This is guaranteed, $\exists t\in S: as' t=a's t$ $\Lrta (f(a)f(s')-f(a')f(s))f(t)=0$ and $f(t)\in \calb^{\times}\Lrta f(a)f(s')-f(a')f(s)=0$
\end{proof}

\begin{ex}
(Most Important Examples)
\begin{itemize}
    \item 
    $\cala\ni f$, $S_f:=\{f^n|n\geq 0\}$ is multiplicatively closed. $\cala_f:=S^{-1}_f\cala$
    \item
    $\scp\subset \cala$ is a prime ideal, then $\cala-\scp$ is multiplicatively closed (By the definition of prime ideals). We can define (In fact then $\cala-\scp$ is multiplicatively closed is equivalent to $\scp$ is prime)
    $\cala_{\scp}:=S_{\scp}^{-1}\cala$
\end{itemize}
Caution that if $\scp=(f)$, usually $\cala_{(f)}\neq \cala_{f}$
\end{ex}

Consider $\varphi:\cala\lrta\calb$ and $\sca\subseteq \cala, \scb\subseteq \calb$. We have defined the extension and contraction of ideals as $\scb^{c}=\varphi^{*}(\sca):=\varphi^{-1}(\scb)$ and $\sca^{c}=\varphi_*(\sca):=\calb\varphi(\sca)$ \textbf{Notice that 
$\scq\subseteq \calb$ prime $\Lrta \varphi^{*}(\scq)$ prime,
thus $\varphi^*:Spec(\calb)\lrta Spec(\cala)$
}.

Back to the special case $\iota :\cala\lrta S^{-1}\cala$.
\begin{prop}\ 
\begin{enumerate}[label=(\alph*)]
\item $\iota^*:Spec(S^{-1}\cala))\lrta\{\scp\in Spec(\cala)|\scp\cap S=\emptyset\}$ is a bijection.
\item For any ideal $\sca\subseteq \cala$, $\iota_*(\sca)=\{a/s|a\in\sca,s\in S\}$
\item $\iota_*(\sca)=S^{-1}(\cala)\Llrta\sca\cap S\neq \emptyset$
\item For any ideal $\scb\subseteq S^{-1}\cala,$ $\varphi_{*}(\varphi^*(\scb))=\scb$
\end{enumerate}
\end{prop}
\begin{proof}\ 
\begin{enumerate}[label=(\alph*)]
    \item $\scq\subseteq S^{-1}\cala$, $\iota^*(\scq)=\iota^{-1}(\scq)$, $\iota(S)\subseteq (S^{-1}\cala)^\times,\Lrta \iota(S)\cap\scq=\emptyset$ otherwise $1\in \scq$ (In fact this part of proof also works for other ideals.)
\item 
Just check that $S^{-1}\cala \cdot V\subset V$
\item
 $\iota_*(\sca)=S^{-1}\cala\Llrta \exists a\in \sca, s\in S$ s.t. $a/s=1/1$
$\Llrta\exists t\in S$ s.t. $\sca\ni ta=ts\in S$, then $\sca\cap S\neq\emptyset$. Conversely, $\sca\cap S\neq \emptyset$, any $a\in \sca, a=s\in S$, then $a/s=1/1$.
\item 
 $\varphi_*(\varphi^*(\scb))\subset \scb$ in general. For the converse inclusion, if $a/s\in \scb$, then $a/s\cdot s/1=a/1\in\scb$, which means $a\in \varphi^{*}(\scb)\Lrta a/s\in \varphi_{*}(\varphi^*(\scb))$. 
\end{enumerate} 
\end{proof}

\subsection{Lecture 8: Properties of localization of rings and localization of module}
Recall
$\iota\cala\lrta S^{-1}\cala$
\begin{itemize}
\item $\iota_*(\sca)=\{\frac{a}{s},a\in \sca,s\in S\}$
\item $\iota_*\iota^*(\scb)=\scb\forall \scb\subseteq S^{-1}\cala$
\item $\iota_*\sca=(1)\Llrta \sca\cap S\neq \emptyset$
\end{itemize}
\begin{prop}
$$
\left\{\scp\in \text{Spec}(\cala)|\scp\cap S=\emptyset\ (S\subseteq \cala-\scp)\right\}\longleftrightarrow \left\{\text{Spec}(S^{-1}\cala)\right\}
$$
is bijection.
\end{prop}

\begin{itemize}
\item $\scq_1\subseteq \scq_2\Llrta \iota^*\scq_1\subseteq \iota^*\scq_2$
\item $k(\scp):=Frac(\cala/\scp)$ is called the residue field of the prime ideal $\scp$.
\end{itemize}
Then the above bijection induces isomorphism $k(\iota^*\scq)\cong k(\scq)$

\begin{ex}
$\cala=\intg$, and $\scp=(p)$ where $p$ is a prime number. $k(\scq)=Frac(\intg/p)=\intg/p$.

If $\scp=(0)$, $k(\scp)=Frac(\intg)=\ratl$.

If $\scp=\scm$ a  maximal ideal.$\Llrta\cala/\scp$ is a field and $k(\scp)=\cala/\scp$ 
\end{ex}

\begin{ex}
$\scp=(y)\subseteq\cala=\cplx[x,y]$, $\cala/\scp\cong \cplx[x], k(\scp)\cong \cplx(x)$
\end{ex}
\begin{proof}
(Proof of the proposition) the proof contains the following points
\begin{itemize}
\item $\scp$ prime $\Llrta \iota^*\scp$ prime
\item $\iota^*\iota_*\scp=\scp$ 
\item  $\iota_*\iota^*\scq=\scq$ (true for any $\scq$, not necessarily prime)
\end{itemize}

$\iota^*\iota_*\scp\supseteq\scp$ is a general fact.
 For the converse inclusion, $\iota^*\iota_*\scp=\iota^{-1}(\iota_*\scp)\overset{?}{\subseteq}\scp$. For an $a\in \iota^{-1}(\iota_*\scp)$, $\iota(a)=\frac{a}{1}\in\iota_*\scp\Lrta\exists b\in \scp, s\in S$ s.t. $\frac{a}{1}=\frac{b}{s}\Lrta ast=bt\in \scp$ and $s,t \in S\subseteq \cala-\scp\Lrta a\in \scp$ because $\scp$ is a prime ideal.

 $\scp$ \text{ prime} $\overset{?}{\Lrta}\iota_*\scp$ prime. Consider $\frac{a}{s}\cdot \frac{b}{t}\in \iota_*\scp$, then $\frac{ab}{st}=\frac{c}{u},c\in\scp, u\in S$, then $\exists v\in S:abuv=cstv$, where $uv\in S$ $cstv\in\scp$, $uv\not\in \scp\Lrta ab\in\scp\Lrta$ at least one of $a, b\in\scp\Lrta$ at least one of $\frac{a}{s},\frac{b}{t}\in \iota_*\scp$.
\end{proof}

\begin{ex}
$S=S_f=\{f^n:n\geq0\}$
$\Lrta S^{-1}\cala=\cala_f=\cala[1/f]$. Let $\scp\cap S\neq \emptyset\Llrta$ some $f^n\in \scp\llrta f\in \scp$.
Then $\text{Spec}(\cala_f)\cong \{\scp\in \text{Spec}(\cala)|f\in \scp\}$
\end{ex}

\begin{ex}
$\cala=\intg,f=2$, $\cala_f=\intg[1/2]$

$\{\text{primes in }\intg[1/2]\}\cong \{(0),(3),(5)....\}\subseteq\text{Spec}(\intg)$
\end{ex}

\begin{ex}
$\cala=\cplx[x,y]$, there is a bijection between $\{\text{maximal ideals in }\cala\}$ and $\cplx^2$.
The maximal ideal $\{f\in \cplx[x,y]| f(X_0,Y_0)=0\}=(x-X_0, y-Y_0)$ corresponds to the point $(X_0,Y_0)\in \cplx^2$\\
Fix $f\in \cplx[x,y], f\neq 0, e.g. f=u-x^2$
Then 
$$
\begin{aligned}
&\left\{ \text{maximal ideals in  }\cala_f=\cplx[x,f,1/f]\right\}\\
&\overset{bij}{\longleftrightarrow} \left\{\text{maximal ideal }\scm\in \cplx[x,y] s.t. f\notin \scm\right\}\\
& \overset{bij}{\longleftrightarrow}\left\{(X,Y)\in \cplx^2|f(X,Y)\neq 0\right\}
\end{aligned}
$$
Then we know that the $\text{Spec}(\cala)\cong \cplx^2$ while $\text{Spec}(\cala_f)$ is bijective to the complement of zero loci of $f$
\end{ex}


The localization at an element has the functorial property, for $f,g\in\cala$
\[
\begin{tikzcd}
\cala\ar[r]\ar[rr,bend right]&\cala_f\ar[r]&\cala_{fg}  
\end{tikzcd}
\] 


\begin{ex}
$\cala$ an integral domain, $\cala_f\subseteq \cala_{fg}$ ($\frac{a}{(f)^n}=\frac{a g^n}{(fg)^n}$), $Frac(\cala)=\cup_f\cala_f$. For any $\scp\in \text{Spec}(\cala_f)\subseteq\text{\cala}$, $\cala_f\Lrta k(\scp)$ 

$\{f\in \cala:f\notin \scp\}=f\in \cala:f(\scp)\neq 0\}$, where $f(\scp)\in k(\scp)$ is the image of $f$.
\end{ex}

\textbf{Aside:} $\cala$ is a local ring  $\Llrta\exists 1\scm\in \text{Specm}(\cala)\Llrta \exists ideal \scm with 1+\scm\subseteq \cala^\times, \scm$ maximal,
$\Llrta \cala-\scm\subseteq \cala^\times$


$\scp\subseteq\cala$ prime $\Lrta \cala_\scp:=S^{-1}_{\scp}\cala$

\begin{prop}
\begin{enumerate}[label=(\alph*)]
\item $\text{Spec}(\cala_\scp)\cong \{\scq\in \text{Spec}(\cala)|\scq\subseteq\scp\}$
\item For $\iota :\cala\lrta S^{-1}_{\scp}\cala$, $\cala_\scp$ is a local ring with maximal ideal $\scp_\scp ?:\iota_*(\scp)$,
\end{enumerate}
$\cala_\scp$ is called the \textbf{localization of $\cala$ at $\scp$}. $\iota_*$ is inclusion preserving.
\end{prop}
\begin{proof}
$$
S^{-1}_{\scp}\overset{\iota_*}{\cong}\left\{\scq\in \text{Spec}(\cala)|\scq\cap S_\scp =\emptyset \ (\scq\subseteq \scp)\right\}
$$.
$\iota_*$ is inclusion preserving,
$\Lrta$ every prime ideal in $\cala_\scp$ is contained in $\scp_\scp$. using this and the fact that any ideal is contained in some maximal ideal, we see that $\scp_\scp\subseteq \cala_\scp$ is the maximal ideal.
\end{proof}

\begin{ex}
$\scp=(p)\subseteq\intg=\cala$, then $\cala_\scp=\intg_{(\scp)}$ is local ring with maximal ideal $\scp_\scp$ generated by image of $\scp$. $\text{Spec}(\intg_{(\scp)})\cong \{\scq \in \text{Spec})(\intg)|\scq\subseteq \scp\}=\{(0),(p)\})$

For residue field $\intg_{(p)}/\scp_\scp\cong \intg/(p)$, this isomorphism is by the first part of the first prop of today's lecture. And in general
$$
\cala_\scp/\scp_\scp=k(\scp)
$$
\end{ex}


\begin{dfn}
A \text{germ at} $p$ is an equivalence class $[(U,f)]$ of pairs $(U,f)$, where $p\in U\subseteq \Omega$ and $f:U\lrta \cplx$ holomorphic. And $(U_1,f_1)\sim (U_2,f_2)$ iff $f_1=f_2$ on some open neighborhood of $p$ inside $U_1\cap U_2$
\end{dfn}
\begin{lemma}
$\Omega\subseteq \cplx$ open $\cala$ is the set of holomorphic germs $f:\Omega\lrta \cplx$. Fix $p\in\Omega$. and set $\scp=\{f\in \cala|f(p)=0\}$. Then $\cala$ is a local ring with maximal ideal $\scp$
\end{lemma}
\begin{proof}
Want $\cala-\scp\subseteq \cala^\times$\\

This is just a way of saying : if $f(p)\neq 0$, then there is an open neighborhood of $p$ on which $1/f$ is defined and holomorphic.
\end{proof}

\begin{ex}
$\cala=\cplx[x,y],\scp=(y)$\\
$$
\text{Spec}(\cala_\scp)\cong \left\{\scq\in \text{Spec}(\cala)|\scq\subseteq (y)\right\}
$$
Then, the only choice of $\scq$ is just $(y),(0)$. 
$\cala_\scp$ is a local ring with two primes, and residue field $\cplx(x)$.

$\cala=\cplx[x,y],\scp=(x,y)$\\
$$
\text{Spec}(\cala_\scp)\cong \left\{\scq\in \text{Spec}(\cala)|\scq\subseteq (x,y)\right\}
$$
Then
$$
\text{Spec}(\cala_\scp)\cong \{(x,y)\}\cup\{(f):0\neq f\in \cplx[x,y] \text{ irreducible }, f(0,0)=0\}\cap\{(0)\}.
$$
The second set is just the set of plane curves passing through $0$
\end{ex}

\subsection*{localization of module}
\begin{dfn}
$S\subseteq \cala$ and $M$ is an $\cala$-module. Then we define the \textbf{localization of module}
$$
(m,s)\in M\times S, (m,s)\sim (m',s')\Llrta \exists t\in S: tsm'=ts'm
$$
and we denote the equivalence class of $(m,s)$ by $\frac{m}{s}$, and we see that $S^{-1}M$ is in fact an $S^{-1}\cala$-module:
$$
\frac{a}{s}\cdot \frac{m}{t}=\frac{am}{st}
$$
\end{dfn}

\begin{lemma}
$S^{-1}\cala\otimes_{\cala}M\cong S^{-1}M$, where the map is $\frac{a}{s}\otimes m\mapsto \frac{am}{s}$
\end{lemma}
\begin{proof}
We can define the inverse
$$
\frac{1}{s}\otimes m \longleftarrow\frac{m}{s}
$$
and then check it is well-defined.
\end{proof}

Moreover, we can also define the localization of morphisms,
\begin{dfn}
Given $f:M\lrta N$ a morphism of $\cala$-module. $S^{-1}$. We define 
$$
\begin{aligned}
S^{-1}f:&S^{-1}M\lrta S^{-1}N\\
& \frac{m}{s}\longmapsto\frac{f(m)}{s}.
\end{aligned}
$$
It is a well-defined morphism of $S^{-1}\cala$-modules and it has the functorial property
$$
S^{-1}(f\circ g)=S^{-1}f\circ S^{-1}g
$$
e.g. 
$\scp\in \text{Spec}(\cala)$, then we have the localization $\cala_\scp$ and the localization of module : $M_\scp:=S^{-1}_\scp M\cong \cala_{\scp}\otimes_\cala M$.
\end{dfn}

Next time: we will focus other local properties i.e.properties of $M$ that depends only on  $M_\scp, \forall \scp\in \text{Spec}(\cala)$








\end{document}