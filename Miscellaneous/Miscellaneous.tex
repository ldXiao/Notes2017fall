\documentclass[11pt]{article}
\usepackage{amssymb}
\usepackage{latexsym}
\usepackage{amsmath}
\usepackage{amsthm}
\usepackage{mathtools}
\usepackage{natbib}
\usepackage{tikz-cd}
\usepackage{enumitem} 
\usepackage{hyperref}
\usepackage{glossaries}
\hypersetup{
    colorlinks,
    citecolor=blue,
    filecolor=blue,
    linkcolor=blue,
    urlcolor=blue
}
\newtheorem{thm}{Theorem}[section]
\newtheorem{prop}[thm]{Proposition}
\newtheorem{lemma}[thm]{Lemma}
\newtheorem{cor}[thm]{Corollary}
\newtheorem{dfn}[thm]{Definition}
\newtheorem{axiom}[thm]{Axiom}
\newtheorem{rmk}[thm]{Remark}
\newtheorem{ex}[thm]{Example}
\newtheorem{question}[thm]{Question}
\newtheorem{problem}[thm]{Problem}
\newtheorem{dfn/thm}[thm]{Definition/Theorem}
\renewcommand{\baselinestretch}{1.05}
\newcommand{\reals}{\mathbb R}
\newcommand{\cplx}{\mathbb C}
\newcommand{\intg}{\mathbb Z}
\newcommand{\bbk}{\mathbb K}
\newcommand{\bbf}{\mathbb F}
\newcommand{\ratl}{\mathbb Q}
\newcommand{\torus}{\mathbb T}
\newcommand{\sca}{{\mathfrak a}}
\newcommand{\scb}{{\mathfrak b}}
\newcommand{\scc}{{\mathfrak c}}
\newcommand{\scm}{{\mathfrak m}}
\newcommand{\scn}{{\mathfrak n}}
\newcommand{\scp}{{\mathfrak p}}
\newcommand{\scq}{\mathfrak q}
\newcommand{\frakg}{{\mathfrak g}}
\newcommand{\frakd}{{\mathfrak d}}
\newcommand{\calf}{{\cal F}}
\newcommand{\calg}{{\cal G}}
\newcommand{\cala}{{\cal A}}
\newcommand{\calb}{{\cal B}}
\newcommand{\calc}{{\cal C}}
\newcommand{\cale}{{\cal E}}
\newcommand{\call}{{\cal L}}
\newcommand{\caln}{{\cal N}}
\newcommand{\calo}{{\cal O}}
\newcommand{\calr}{{\cal R}}
\newcommand{\mathbold}{\bf}
\newcommand{\cinf}{C^{\infty}}
\newcommand{\row}[2]{#1_1,\dots ,#1_{#2}}
\newcommand{\dbyd}[2]{{\partial #1\over\partial #2}}
\newcommand{\Space}{{\bf Space}}
\newcommand{\alg}{{\mathbold Alg}}
\newcommand{\notsubset}{\not \subset}
\newcommand{\notsupset}{\not \supset}
\newcommand{\pois}{{\mathbold Pois}}
\newcommand{\pitilde}{\tilde{\pi}}
\newcommand{\rta}{\rightarrow}
\newcommand{\Lrta}{\Longrightarrow}
\newcommand{\lrta}{\longrightarrow}
\newcommand{\llrta}{\longleftrightarrow}
\newcommand{\Llta}{\Longleftarrow}
\newcommand{\Llrta}{\Longleftrightarrow}
\newcommand{\lgl}{\langle}
\newcommand{\rgl}{\rangle}
\newcommand{\inj}{\hookrightarrow}
\newcommand{\downmapsto}{\rotatebox[origin=c]{-90}{$\scriptstyle\mapsto$}\mkern2mu}
\renewcommand{\qedsymbol}{$\square$}
\bibliographystyle{plain}
\title{\bf Personal Notes All the information I need}
\author{Texed by Lin-Da Xiao} %\thanks{Research partially supported by NSF Grant DMS-96-25122 and the Miller Institute for Basic Research in Science.}
\begin{document}
\maketitle
\tableofcontents
\newpage
\section{Affine Group Scheme, Hopf Algebra}
\begin{dfn}
Let $A$ be a commutative $K$-algebra. The corresponding affine $K$-scheme $G=\text{Spec}(A)$ is said to be a group scheme if it is endowed with algebraic operations
$$
\begin{aligned}
&\mu:G\times G\lrta G\ (\text{product}),\\
&e: \text{Spec}(K)\lrta G (\text{unit}),\\
& \iota: G\lrta G (\text{inverse}),
\end{aligned}
$$
satisfying the usual axioms of a group, which are expressed by the commutativity of the following three diagrams
\begin{enumerate}[label=(\arabic*)]
\item Associativity:
    \[
    \begin{tikzcd}
    G\times G\times G\ar[r,"\mu\times Id"]\ar[d,"Id\times \mu"]& G\times G\ar[d,"\mu"]\\
    G\times G\ar[r,swap,"\mu"] & G
    \end{tikzcd}
    \]

\item Unit:
    \[
    \begin{tikzcd}
    G\times Spec(K)\ar[r,"Id\times e"]\ar[dr,"pr_1"]& G\times G\ar[d,"\mu"]& Spec(K)\times G\ar[l,swap,"e\times Id"]\ar[dl,"pr_2"]\\
     & G &
    \end{tikzcd}
    \]
\item Inverse:
    \[
    \begin{tikzcd}
    & G\times G\ar[dr,"\mu"]& \\
    G\ar[r,"\pi"]\ar[ur,"Id\times\iota"]\ar[dr,"\iota\times Id"]& Spec(K)\ar[r,"e"] &G\\
     & G\times G \ar[ur,"\mu"]&
    \end{tikzcd},
    \]    
    where $\pi $ denotes the structural map of $G$ as a $K$-scheme. If the algebra $A$ is finitely generated, we say that $G$ is algebraic. We will see below that every affine group scheme is in fact a projective limit of affine group schemes. 
\end{enumerate}
\end{dfn}
{\color{red}
Equivalent definition: A group scheme over $K$ is a functor between the categories of commutative $K$-algebras and abstract groups. Namely, given $G=\text{Spec}(A)$ $A,P,Q$ are $K$-algebras, one considers the functor:
    \[
    \begin{tikzcd}
    P\ar[r,"f"]\ar[d,"G"]& Q\ar[d,"G"] \\
    Hom_{K-alg}(A,P)=G(P)\ar[r,"g\mapsto G(g)=f\circ g"] & G(Q)=Hom_{K-alg}(A,Q)
    \end{tikzcd},
    \]}
\subsection{Hopf Algebra}
The category of affine schemes over $K$ is equivalent to the category of the commutative $K$-algebras through the contravariant functors
$$
A\mapsto \text{Spec}(A)
$$
$$
G\mapsto \calo(G),
$$
where $\calo(G)$ is the ring of regular functions on $G$. Thus the defining properties of a group scheme can be transfered to the corresponding algebra, yielding the concept of Hopf algebra.

\begin{dfn}
Let $H$ be an associative (not necessarily commutative) $K$-algebra. Let $\nabla:H\otimes H\lrta H$ be the product of $H$ and $\eta :k\lrta H$ the unit.
\begin{enumerate}[label=(\arabic*)]
\item We say that  $H$ is a \textbf{bialgebra} if it is provided with two morphisms of algebras
$$
\begin{aligned}
& \Delta:H\lrta H\otimes H(\text{coproduct}),\\
& \epsilon: H\lrta k(\text{counit})
\end{aligned}
$$
such that the following diagrams commute:
\begin{enumerate}[label=(\alph*)]
\item Coassociativity:
    \[
    \begin{tikzcd}
    H\ar[r,"\Delta"]\ar[d," \Delta"]& H\otimes H\ar[d,"Id\otimes \Delta"] \\
    H\otimes H\ar[r,"\Delta\otimes Id"] & H\otimes H\otimes H,
    \end{tikzcd}
    \]
\item Counit.
    \[
    \begin{tikzcd}
    H\otimes K & H\otimes H\ar[l,"Id\otimes\epsilon"] \ar[r,"\epsilon\otimes Id"] & K\otimes H \\
    & H\ar[ul]\ar[ur]\ar[u,"\Delta"] &
    \end{tikzcd}
    \]
\end{enumerate}
\item A bialgebra $H$ is called a \textbf{Hopf algebra} if it is further equipped witha morphism of algebras
$$
S:H\lrta H \ (\text{antipode})
$$
such that the following diagram commutes:
\begin{enumerate}[label=(c)]
\item Antipode.
    \[
    \begin{tikzcd}
        & H\otimes H\ar[rr,"S\otimes Id"] &\ & H\otimes H\ar[dr,"\nabla"] & \\
    H\ar[rr,"\epsilon"]\ar[ur,"\Delta"]\ar[dr,"\Delta"]& & k\ar[rr,"\eta"] & & H\\
    & H\otimes H\ar[rr,"Id\otimes S"] & & H\otimes H\ar[ur, "\nabla"] &
    \end{tikzcd}
    \]
\end{enumerate}
\item A bialgebra $H$ is called commutative if the product is commutative, and cocommutative if the coproduct satisfies $\Delta=\tau\circ \Delta$, where $\tau:H\otimes H\lrta H\otimes H$ is the flip of the factors.
\end{enumerate}
\end{dfn}

\section{Tannakian Categories}
$K$ is a field of characteristic 0, $\cala$ rigid tensor $\bbk$-linear abelian category, $L$ is an extension of $K$.
\begin{dfn}
An $L$-valued \textbf{fibre functor }is a tensor functor $\omega:\cala\lrta Vec_L$ which is faithful and exact.
\end{dfn}
\begin{dfn}
$\cala$ is 
\begin{itemize}
    \item \textbf{Neutralized Tannakian} if one can  endow it with a $K$-valued fibre functor.
    \item \textbf{Neutral Tannakian} if $\exists$ $K$-valued fibre functor.
    \item \textbf{Tannakian } if $\exists L$-valued fibre functor for some $L$.
\end{itemize}
\end{dfn}
\begin{ex}
$G$ is an affine $K$-group scheme, $\cala=Rep_K(G),\omega:\cala\lrta Vec_K$ the forgetful functor.
\end{ex}










\end{document}