\documentclass[11pt]{article}
\usepackage{amssymb}
\usepackage{latexsym}
\usepackage{amsmath}
\usepackage{amsthm}
\usepackage{mathtools}
\usepackage{natbib}
\usepackage{tikz-cd}
\usepackage{enumitem} 
\usepackage{hyperref}
\hypersetup{
    colorlinks,
    citecolor=blue,
    filecolor=red,
    linkcolor=blue,
    urlcolor=blue
}
\newtheorem{thm}{Theorem}[section]
\newtheorem{prop}[thm]{Proposition}
\newtheorem{lemma}[thm]{Lemma}
\newtheorem{exercise}[thm]{Exercise}
\newtheorem{cor}[thm]{Corollary}
\newtheorem{dfn}[thm]{Definition}
\newtheorem{axiom}[thm]{Axiom}
\newtheorem{rmk}[thm]{Remark}
\newtheorem{ex}[thm]{Example}
\newtheorem{question}[thm]{Question}
\newtheorem{problem}[thm]{Problem}
\renewcommand{\baselinestretch}{1.05}
\newcommand{\pd}{\partial}
\newcommand{\reals}{\mathbb R}
\newcommand{\cplx}{\mathbb C}
\newcommand{\intg}{\mathbb Z}
\newcommand{\bbf}{\mathbb F}
\newcommand{\bbk}{\mathbb K}
\newcommand{\ratl}{\mathbb Q}
\newcommand{\torus}{\mathbb T}
\newcommand{\sca}{{\mathfrak a}}
\newcommand{\scb}{{\mathfrak b}}
\newcommand{\scc}{{\mathfrak c}}
\newcommand{\scm}{{\mathfrak m}}
\newcommand{\scn}{{\mathfrak n}}
\newcommand{\scp}{{\mathfrak p}}
\newcommand{\frakg}{{\mathfrak g}}
\newcommand{\frakd}{{\mathfrak d}}
\newcommand{\calf}{{\cal F}}
\newcommand{\calg}{{\cal G}}
\newcommand{\cala}{{\cal A}}
\newcommand{\calc}{{\cal C}}
\newcommand{\cale}{{\cal E}}
\newcommand{\calk}{{\cal K}}
\newcommand{\call}{{\cal L}}
\newcommand{\caln}{{\cal N}}
\newcommand{\calm}{{\cal M}}
\newcommand{\calo}{{\cal O}}
\newcommand{\calr}{{\cal R}}
\newcommand{\mathbold}{\bf}
\newcommand{\cinf}{C^{\infty}}
\newcommand{\row}[2]{#1_1,\dots ,#1_{#2}}
\newcommand{\dbyd}[2]{{\pd #1\over\pd #2}}
\newcommand{\Space}{{\bf Space}}
\newcommand{\alg}{{\mathbold Alg}}
\newcommand{\notsubset}{\not \subset}
\newcommand{\notsupset}{\not \supset}
\newcommand{\pois}{{\mathbold Pois}}
\newcommand{\pitilde}{\tilde{\pi}}
\renewcommand{\qedsymbol}{$\square$}
\newcommand{\rta}{\rightarrow}
\newcommand{\Lrta}{\Longrightarrow}
\newcommand{\lrta}{\longrightarrow}
\newcommand{\Llta}{\Longleftarrow}
\newcommand{\lgl}{\langle}
\newcommand{\rgl}{\rangle}
\newcommand{\inj}{\hookrightarrow}
\bibliographystyle{plain}
\title{\bf Notes for Differential Galois Theory by P. Jossen}
\author{Texed by Lin-Da Xiao} %\thanks{Research partially supported by NSF Grant DMS-96-25122 and the Miller Institute for Basic Research in Science.}
\begin{document}
\maketitle
\tableofcontents
\newpage
\section{The Ring of Partial Differential Operators}
\begin{dfn}
A $\Delta$-ring $\calr$ is a commutative ring with unit equipped with a set of commutative derivations $\Delta=\{\pd_1,...,\pd_r\}$. A $\Delta$-ideal $I\subset \calr$ is an ideal of $\calr$ s.t.  $\pd_i I\subset I$ for all $i=1,...,r$. If $\calr$ is a $\Delta$-ring, the set of constants is $\{c\in \calr|\pd_i(c)=0, \forall i\}$. Through out the expository article, we will assume that for any $\Delta$-ring, $\ratl\subset \calr$ and $\calc$ is algebraically closed field.
\end{dfn}
\begin{ex}
$\calc$ be an algebraically closed with $t_1,...,t_r$ indeterminates. And the derivations are define to by $\pd_it_j=\delta_{ij}$ 
\end{ex}

\begin{dfn}
Let $\calk$ be a $\Delta$-field. The ring of partial differential operators $\calk[\pd_1,...,\pd_r]$ with coefficients in $\calk$ is the noncommutative polynomial ring in the variables $\pd_i$, where $[\pd_i,\pd_j]=0$ but $[\pd_i, a]=\pd_i(a)$
\end{dfn}

\begin{dfn}
A $\calk[\pd_1,...,\pd_r]$-module $\calm$ is a finite dimensional $\calk$-vector space that is a left module for the ring $\calk[\pd_1,...,\pd_r]$.
\end{dfn}

But in this case a quotient $\calr/I$ of $\Delta$





\end{document}
