\documentclass[11pt]{article}
\usepackage{amssymb}
\usepackage{latexsym}
\usepackage{amsmath}
\usepackage{amsthm}
\usepackage{mathtools}
\usepackage{natbib}
\usepackage{tikz-cd}
\usepackage{enumitem} 
\usepackage{hyperref}
\usepackage{bbm}
\hypersetup{
    colorlinks,
    citecolor=blue,
    filecolor=red,
    linkcolor=blue,
    urlcolor=blue
}
\newtheorem{thm}{Theorem}[section]
\newtheorem{prop}[thm]{Proposition}
\newtheorem{lemma}[thm]{Lemma}
\newtheorem{exercise}[thm]{Exercise}
\newtheorem{cor}[thm]{Corollary}
\newtheorem{dfn}[thm]{Definition}
\newtheorem{axiom}[thm]{Axiom}
\newtheorem{rmk}[thm]{Remark}
\newtheorem{ex}[thm]{Example}
\newtheorem{question}[thm]{Question}
\newtheorem{problem}[thm]{Problem}
\renewcommand{\baselinestretch}{1.05}
\newcommand{\pd}{\partial}
\newcommand{\reals}{\mathbb R}
\newcommand{\cplx}{\mathbb C}
\newcommand{\intg}{\mathbb Z}
\newcommand{\bbf}{\mathbb F}
\newcommand{\bbk}{\mathbb K}
\newcommand{\ratl}{\mathbb Q}
\newcommand{\torus}{\mathbb T}
\newcommand{\sca}{{\mathfrak a}}
\newcommand{\scb}{{\mathfrak b}}
\newcommand{\scc}{{\mathfrak c}}
\newcommand{\scm}{{\mathfrak m}}
\newcommand{\scn}{{\mathfrak n}}
\newcommand{\scp}{{\mathfrak p}}
\newcommand{\frakg}{{\mathfrak g}}
\newcommand{\frakd}{{\mathfrak d}}
\newcommand{\calf}{{\cal F}}
\newcommand{\calg}{{\cal G}}
\newcommand{\cala}{{\cal A}}
\newcommand{\calc}{{\cal C}}
\newcommand{\cale}{{\cal E}}
\newcommand{\calk}{{\cal K}}
\newcommand{\call}{{\cal L}}
\newcommand{\caln}{{\cal N}}
\newcommand{\calo}{{\cal O}}
\newcommand{\calr}{{\cal R}}
\newcommand{\mathbold}{\bf}
\newcommand{\cinf}{C^{\infty}}
\newcommand{\row}[2]{#1_1,\dots ,#1_{#2}}
\newcommand{\dbyd}[2]{{\pd #1\over\pd #2}}
\newcommand{\Space}{{\bf Space}}
\newcommand{\alg}{{\mathbold Alg}}
\newcommand{\notsubset}{\not \subset}
\newcommand{\notsupset}{\not \supset}
\newcommand{\pois}{{\mathbold Pois}}
\newcommand{\pitilde}{\tilde{\pi}}
\renewcommand{\qedsymbol}{$\square$}
\newcommand{\rta}{\rightarrow}
\newcommand{\Lrta}{\Longrightarrow}
\newcommand{\lrta}{\longrightarrow}
\newcommand{\Llta}{\Longleftarrow}
\newcommand{\lgl}{\langle}
\newcommand{\rgl}{\rangle}
\newcommand{\inj}{\hookrightarrow}
\bibliographystyle{plain}
\title{\bf Notes for Differential Galois Theory by P. Jossen}
\author{Texed by Lin-Da Xiao} %\thanks{Research partially supported by NSF Grant DMS-96-25122 and the Miller Institute for Basic Research in Science.}
\begin{document}
\maketitle
\tableofcontents
\newpage
\section*{About the Course}
Classical Galois theory:
\begin{itemize}
\item Fields $\calk$
\item 
 Fields extensions $\calk'/\calk$.
 \item 
$Gal(\calk'/\calk)= Aut_\calk(\calk')$.
\item
Solution of polynomials, $\calk'$=``$\calk$(solutions of polynomials)'', which is what we call \textbf{splitting field}.
\end{itemize}
 
\noindent In the Differential setting, 
\begin{itemize}
\item Fields $\calk$ with derivation $\pd$, e.g. $\ratl(t)$, with usual derivative:
\item Differential field extension $\calk'\calk$
\item $Gal(\calk'/\calk)=Aut_{(\calk,\pd)}(\calk',\pd)$
\item Solution of differential equations. $\calk'=``\calk(\text{solution of differential equation})''$, which is called \textbf{Picard-Vessiot Field}
\end{itemize}

\section{Differential rings and modules}
Convention: Ring= commutative ring with unit.
\begin{dfn}
Let $\calr$ be a ring. A \textbf{derivation} on  $\calr$ is a map $\pd:\calr \rightarrow \calr$ s.t.
\begin{enumerate}
\item $\pd (a+b)=\pd a+\pd b$,
\item $\pd(ab)=a\pd b+b\pd a$.
\end{enumerate}
then we call $(\calr,\pd)$ a differential ring.
A morphism of diff-rings $\varphi:(\calr_1,\pd_1 )\rightarrow (\calr_2,\pd_2)$ is a ring morphism s.t. $\pd_2\varphi=\varphi\pd_1$
\end{dfn} 
\begin{ex}
$\ratl(t),\calk(t)$ with usual derivations and $\ratl[t],\calk[t]$, $C^\infty([0,1])$
\end{ex}

\begin{dfn}
Let $(\calr, \pd)$ be a diff-ring, Call $\calc\subset\calr$ \textbf{constant} it $\pd c=0,\forall c\in \calc$.
\end{dfn}

\begin{prop}
Let $\calc\subset\calr$ be the set of constant elements.
\begin{enumerate}
\item $\calc$ is a subring,
\item If $\calr$ is a field, then so is $\calc$
\end{enumerate}
\end{prop}
\begin{proof}
$1\in\calc$, $\pd 1=\pd (1\cdot 1)=2\pd 1\Longrightarrow \pd c=0$\\
$a,b\in \calc\Longrightarrow a+b\in \calc$, and $ab\in\calc$ by Leibnitz rule.\\
Suppose $\calr$ is a field ,$c\in \calc, c\neq 0, 0=\pd 1=\pd (c\cdot c^{-1})=c\pd (c^{-1})\Longrightarrow \pd c^{-1}=0$
\end{proof}
\begin{ex}
Caution: $\calr=\bbf_p[t]$, $\pd$ is the usual derivative. Here, constant $\calc= \bbf_p[t^p]\supsetneqq \bbf_p$, because
$$
\pd(a_0+a_1 t^p+a_2 t^{2p}+...+a_n t^{np})=0.
$$
\end{ex}

\begin{exercise}
Show that $\calc \subseteq \calr$ is a algebraically closed in $\calr$. i.e. $x\in \calr$ algebraic over $\calc$ $\Longrightarrow$ $x\in \calc$. (Notice it does not mean $\calc=\overline{\calc}$ in general)
\end{exercise}

\begin{exercise}\label{exr:1.7}
Show that for differential field $\calk$ with constants $\calc$, consider a field extension $\calr/\calk$, an element $x\in \calr$ satisfying $x'=0\in \calk$  is algebraic over $\calk\Lrta x$ is algebraic over $\calc$.\\
\textbf{Solution}:\\
$x$ is algebraic over $\calk$ , consider the minimal monic polynomial $p(X)= X^n+...a_0$ with coefficients in $\calk$. Then $p(c)=0\Lrta p(x)'=(a_{n-1}')x^{n-1}+...+(a_0')=0$ by the minimality of $p(X)$, we conclude that we conclude that each $a_{i}'=0$, thus finished the proof.
\end{exercise}

\begin{dfn}
A \textbf{differential $(\calr, \pd)$-module} $(M,\pd)$ is a $\calr$-module  $M$ , together with $\pd: M\rightarrow M $ satisfying:
\begin{enumerate}
\item $\pd(m+n)=\pd m+\pd n$
\item $\pd_M(a m)=\pd_\calr a \cdot m+a\cdot \pd_M m$.
\end{enumerate}
Think of $(M,\pd)$ as a differential equation, with solutions $ker(\pd :M\rightarrow M).$
\end{dfn}
Suppose $\calr=\calk$ is a field (over that $M$ is free), $M $ has finite dimension.
Choose a $\calk$-basis , $(e_1,...,e_n)$ of $M$.
Set
$$
\pd e_i=-\sum_{j=1}^n a_{ij} e_j,
$$
where $A=(a_{ij})\in  M_{n \times n}(\calk)$. 
The matrix $A$ characterizes $\pd :M\rightarrow M$, uniquely, by additivity and Leibnitz:
$$
m\in M, m=\sum_{i=1}^n \lambda_i e_i,
$$

$$
\begin{aligned}
\pd m & =\sum \pd (\lambda_i e_i)=\sum (\pd \lambda_i)e_i+\sum \lambda_i \pd e_i\\
&=\sum(\pd \lambda_i) e_i-\sum \sum \lambda_i a_{ij} e_j.
\end{aligned}
$$
The differential equation corresponding to $(M,\pd)$ is  the equation
$$
u'=A u.
$$
\begin{rmk}
the matrix $A$ depends on the choice of the $\calk$-basis of $M$. Choosing a different basis yields an equation $u'=\tilde{A}u$ with $\tilde{A}=S^{-1}A S-S^{-1}S'$, where $S\in GL_n(\calk)$ is the base change matrix. we called $A,\tilde{A}$ equivalent
\end{rmk}
\begin{rmk}
Let $M$ be a differential $\calk$-module, $\calc\subset \calk$ be the set of constants, then we have
$
\pd  :M\rightarrow M.
$ 
is $\calc$-linear, follows from Leibnitz.
In particular $ker \pd\subseteq M$ is a $\calc$-module (vector space)
\end{rmk}

\begin{lemma}\label{lem:basis}
Let $u'=A u$ be a differential equation with $A\in M_{n\times n}(\calk)$. Let $v_1,...,v_r\in \calk^n$ be solutions, i.e. $v_i'=A v_i$. If $v_1,...,v_r$ are linear dependent over $\calk$, then they are linear dependent over $\calc$. In particular,
$$
dim_{\calc}(\ker \pd)\leq n.
$$
\end{lemma}
\begin{proof}
Induction on $r$. For $r=1$, trivial. Fix $r\geq 2$, suppose lemma holds for $<r$ solutions. Suppose w.l.o.g that no proper subset of $\{v_1,...,v_r\}$ is linear dependent over $\calk$. We find that there is a unique linear dependence relation 
$$
v_1=\sum_{i=2}^r b_i v_i,\ b_i\in \calk
$$
$$
\begin{aligned}
0=v_1'-A v_1&=\sum b_i'v_i+b_i v_i' -\sum_{i=2}^r b_i A v_i\\
& =\sum^r_{i=2}b_i' v_i+\sum_{i=2}^r b_i(v_i'-A v_i)=\sum_{i=2}
^r b_i' v_i
\end{aligned}
$$
so $b_i'=0$ for $i=2,...,r\Longrightarrow b_i\in\calc$, and $v_1,...,v_r$ linear dependent over $\calc$.
\end{proof}
Compactify the notation , $v_1,...,v_r$ columns of a matrix $V\in M_{n\times r}(\calk)$, then $v_i'=A v_i\Longrightarrow V'=A V$. Know that $rank_\calc V\leq n$.
What we usually seek is a $V\in GL_n(\calk)$ with $V'=A V$. The columns of such $V $ provide a basis of the solution space of the differential equation and thus also a basis of the differential module itself.

\begin{dfn}
$\calk$ is a diff-field, $A\in M_{n\times n}(\calk)$, let $\calr$ be a differential $\calk$-algebra, which means we have a diff-ring morphism from $(\calr,\pd_\calr)$ to $(\calk,\pd_\calk)$. We also suppose $\calr$ has same constants as $\calk$ (every constant of $\calr$ lies in $\calk$).
A matrix $V\in GL_n(\calr)$ is said to be a \textbf{fundamental matrix} of solutions of the differential equation $u'=A u$, if $V'=A V$.
\end{dfn}
\begin{rmk}\label{rmk:fundamental_matrix}
Let $V,\tilde{V}$ be fundamental matrices of solutions of $u'=A u$, $V,\tilde{V}\in GL_{n}(\calr)$, $\tilde{V} =V\cdot S$ $S=V^{-1} \tilde{V}$
$$
A\tilde{V}=\tilde{V}'=(VS)'=V'S+V S'=A V S+V S'=A \tilde{V}+V S'
$$
$
\Longrightarrow V S'=0,
$
$V$ is invertible $\Longrightarrow S'=0\Longrightarrow S\in GL_n(C)$.
\end{rmk}

\begin{dfn}
Let $v_1,...,v_n\in \calk$, The \textbf{Wronski matrix} of $\underline{v}\in \calk^n$
is 
$$
Wr(\underline{v})=
\begin{pmatrix}
v_1  & v_2 & \cdots & v_n\\
v_1^{(1)} & v_2^{(1)} & \cdots & v_n^{(1)}\\
\vdots  & \vdots & \ddots & \vdots \\
v_1^{(n-1)} & v_2^{(n-1)} & \cdots & v_n^{(n-1)}
\end{pmatrix}
$$
the \textbf{Wronskian} is the determinant of the Wronski matrix, i.e. $det(Wr(\underline{v}))=:wr(\underline{v})$.
\end{dfn}

-----------------------------personal notes
\begin{dfn}
 A \textbf{Picard-Vessiot ring} over $\calk$ for the equation $v'=A y$, with $A\in M_{n\times n}(\calk)$, is a differential ring $\calr$ over $\calk$, such that:
\begin{enumerate}
\item $\calr$ is a simple differential ring (whose only differential ideal is $(0)$).
\item There exists a fundamental matrix $F$ for $v'=A v$, with coefficients in $R$.
\item $\calr$ is generated as a algebra over $\calk$ by the entries of the fundamental matrix $F$ and the inverse of the determinant of $F$, i.e. $\calr=\calk\langle(F_{ij}), 1/det(F)\rangle$.
\end{enumerate}
A \textbf{Picard-Vessiot ring} for a differential module $M$ over $\calk$ is defined as the Picard-Vessiot ring of a matrix differential equation $v'=A v$ associated $M$. One can check that the definition does not depend on the choice of the representing differential equation of the differential module.
\end{dfn}


\begin{prop}
Let $M$ by a differential module over  $\calk$ of dimension $n$. We have the following equivalent definition of Picard-Vessiot ring of the diff-module $M$.
\begin{enumerate}[label=(\alph*)]
\item $\calr$ is a simple diff-ring.
\item $V:=ker(\pd, \calr\otimes_{\calk} M)$ has dimension $n$ over $\calc$.
\item Let $e_1,...,e_n$ denote any basis of $M$ over $\calk$, then $\calr$ is generated over $\calk$ by the coefficients of all $v\in V$, w.r.t the basis $e_1,...,e_n$ of $\calr\otimes_\calk M$.
\end{enumerate}
\end{prop}
\begin{proof}

\end{proof}
$(b)+(c)\Longrightarrow (2)+(3)$: If the solution space of $\pd$ on $\calr\otimes_\calk M$ has dimension $n$, then we can find a basis $v_1,...,v_n$ over $\calc$, with each $v_i'=A v_i$. By \ref{lem:basis}, we know $v_1,...,v_n$ is linear independent over $\calk,\calr$. Then $\tilde{F}=(v_1,...,v_n)$ is already a fundamental matrix in $GL_n(\calr)$. If $\calr$ is generated by the coefficients of all $v\in V$, in particular, it is generated by a set that contain the entries of $\tilde{F}$ above. On the other hand $V$ is spanned by the column matrices of $\tilde{F}$over $\calc$. $\tilde{F}\in GL_n(\calr)\Longrightarrow \tilde{F}^{-1}\in GL_n(\calr)\Longrightarrow det(\tilde{F})^{-1}\in R$.
-----------------------------



\section{Picard-Vessiot extension}

Through out this section, we will assume 
$\calk$ is a differential field with $char(\calk)=0$(It contains $\ratl$ as subfield). The set of constants $\calc\subset \calk$ is a field and we assume it to be algebraic closed.  For example, think of $\calk=\cplx(t),\pd=d/dt$ and $\calc=\cplx$.

\begin{dfn}
Let $\calr$ be a diff. $\calk$-algebra. An ideal $I\subset \calr$ is a \textbf{differential ideal} if $\pd I\subseteq I$. Say that $\calr$ is \textbf{simple} if $\{0\}$ and $\calr$ are the only differential ideal in $\calr$
\end{dfn}

\begin{rmk}
$I\subseteq\calr$ is differential ideal, then the derivation on $\calr$ induces a derivation on $\calr/I$. Given any morphism of diff. rings $\varphi:\calr\lrta \calr'$, then $Ker(\varphi)$ is a differential ideal.
\end{rmk}

\begin{dfn}
Let $A\in M_n(\calk)$, consider the matrix differential equation
$$
u'=Au
$$

A differential $\calk$-algebra $\calr$ is said to be a \textbf{Picard-Vessiot extension} for  $u'=Au$ if
\begin{enumerate}
\item $\calr$ is simple 
\item The equation $u'=Au$ admits a fundamental matrix of solution in $\calr$, i.e. $\exists V\in M_n(\calr)$, invertible such that
$$
V'=AV.
$$
\item As $\calk$-algebra, $\calr$ is generated by the coefficients  $v_{ij}$ of $V$ and $det(V)^{-1}$.
\end{enumerate}
\end{dfn}
Some references require in addition the constants of $\calr$ are $\calc$. We will see this additional requirement can be derived by 1-3 in our setting.

A \textbf{Picard-Vessiot extension of a differential module $M$} is a Picard-Vessiot extension for any of the corresponding matrix differential equation. Exercise: check this.

Alternatively, we can define the Picard-Vessiot extension of a differential module $M$ directly. Given a diff. module $(M,\pd_M)$, a Picard-Vessiot extension for $(M,\pd_M)$ is a diff. $\calk$-algebra $\calr$ s.t.
\begin{enumerate}
\item $\calr$ is simple
\item $dim(\pd_{\calr\otimes M})=\dim_\calk M$, where $\pd_{\calr\otimes M}:\calr\otimes M\lrta \calr\otimes M $, $\pd_{\calr\otimes M}(r\otimes m)=r'\otimes m+r\otimes \pd_M m$. 
\item $\calr$ is minimal with these properties.
\end{enumerate}

Exercise: Check that the two definitions of PV extension for diff. module coincide.

\begin{ex}
$\calk=\cplx(t), \calc=\cplx$, Consider the 2nd order homogeneous linear differential equation
$$
t\cdot u''+u'=0.
$$
We can set $v:=u', v'=u''$, then we have a new 1st order 
$$
\begin{pmatrix}
u'\\
v'
\end{pmatrix} =\begin{pmatrix}
0 & 1\\
0 & -1/t
\end{pmatrix}\begin{pmatrix}
u\\
v
\end{pmatrix},
$$
where the matrix is called \textbf{companion matrix}.
What we want is $\calr\supseteq \cplx(t), V\in GL_2(\calr)$ s.t. $V'=AV$. The general solution of the 2nd order equation is 
$$
a+b \log(t).
$$
Solution to $(u',v')^T=A\cdot(u,v)^T$ are
$$
\begin{aligned}
\begin{pmatrix}
1\\
0
\end{pmatrix},
&
\begin{pmatrix}
\log(t)\\
1/t
\end{pmatrix},
\end{aligned}
$$
The corresponding fundamental matrix 
$$
V=\begin{pmatrix}
1 & \log(t)\\
0 & 1/t
\end{pmatrix}.
$$
A candidate of the Picard-Vessiot ring is 
$$
\calr=\cplx(t)[X]
$$
the Differential on $\calr$, we just need to set $\pd X=X'=1/t$, in this case, we don't have to adjoint $det(V)^{-1}$. The only thing ledt to check is whether $\calr$ is simple.

Yes, $I\subseteq \calr$ diff. ideal. $\calr$ principal $I=f(X)\calr$,  $\pd I\subset I$, where $\pd I=(\pd f)\calr$, derive it sufficiently many times. until $\pd^n f\in \cplx(t)$ $\Lrta \calr=\pd^n f\calr \subseteq I$.  
\end{ex}

\begin{lemma}\label{lem:PV}
Let $\calr$ be simple diff. $\calk$-algebra. Then 
\begin{enumerate}
\item $\calr$  is integral domain
\item Suppose $\calr$ is finitely generated as $\calk$-algebra, then $Frac(\calr)=\call$ is a differential field with constants equal to $\calc$. 
\end{enumerate}
\end{lemma}
\begin{proof}
The proof of the second part relies on the assumption that $\calc$ is algebraic closed, we will postpone it a little.

Proof of (1) Pick $a\in \calr$ and $a\neq0$. Consider the ideal $I=\{b\in \calr|a^n\cdot b=0, \text{ for some $n\geq1$}\}$. This is a differential ideal. (Check it by derive it and multiply it with $a$)
If $a$ is not nilpotent, then $1\not\in I$ then $I$ has to be $0$, so $a$ is not a zero divisor.
It follows that if $a$ is a zero divisor, then $a$ has to be nilpotent.

Let $I\subset \calr$ be the nil radical of $\calr$. Again, $I$ is a diff. ideal. But $1\not \in I$, $I$ has to be $\{0\}$.
\end{proof}

\begin{lemma}(Criterion for Algebraicity)
Let $\calk$ be a field of $\text{char } 0$. $\calr$ is a finitely generated $\calk$-algebra, integral domain, and suppose that $x\in \calr$ is such that the set $S:=\{c\in \calk|x-c\in \calr^{\times}\}$ is infinite. Then $x$ is algebraic over $\calk$.
\end{lemma}
For example $\calk=\bar{\ratl}, \ \calr=\bar{\ratl}[x,y,1/x,1/(x^2+2)]$. For $x-c$ to be unit, then only possibility is  $c=0$, which means $x$ is not algebraic over $\bar{\ratl}$. While in the case
$\calk=\ratl, \ \calr=\ratl[\sqrt{2}]$, $\sqrt{2}-c$ is always unit in $\calr^\times$, which means $\sqrt{2}$ is algebraic over $\ratl$.
\begin{proof}
Say $\calr=\calk[x_1,...,x_n]$ and w.l.o.g. $x_1=x$. Set $\call=Frac(\calr)$ ($\calr$ is integral domain), suppose $x$ is \textbf{not} algebraic over $\calk$, so $x$ is transcendental. Suppose w.l.o.g. that
Reorder $x_2,...,x_n$ such that  $x_1,...,x_r$ are a transcendence base for $\call/\calk$, i.e. $x_1,..,x_r$ are algebraic independent and $L/\calk(x_1,...,x_r)$ is a finite algebraic extension. Recall the \href{https://en.wikipedia.org/wiki/Primitive_element_theorem}{Lemma of primitive element} (every finite field extension of char $0$, can be generated by one element). Pick $y\in \calr$ such that $\call=\calk(x_1,...,x_r)[y]$, look at the minimal polynomial of $y$ over $\calk[x_1,...,x_r]$
$$
a_N(x_1,...,x_r)T^N+a_{N-1}(x_1,...,x_r)T^{N-1}+...,
$$
where $a_i\in \calk[x_1,...,x_r]$

Pick $G\in \calk[x_1,...,x_r]$ s.t.
\begin{enumerate}[label=\arabic*)]
\item $a_N|G$ and
\item $x_1,...,x_n\in \calk[x_1,...,x_r,y,G^{-1}]$
\end{enumerate} 
For $s>r$, $x_s\in \calr\subseteq \call=Frac(\calk[x_1,...,x_r])[y]$
$$
x_s=\frac{P_s(x_1,...,x_r,y)}{Q_s(x_1,...,x_r)}=\frac{\tilde{P}_s(x_1,...,x_r,y)}{G}
$$
$G$ has to be a multiple of all those denominators $Q_s$, it is always possible to pick such a $G$.

Since the set $S\subseteq \calk$ is infinite, we can find $s_1,...,s_r\in S$ with $G(s_1,...,s_r)$ $\neq 0$. Fix such elements $s_1,...,s_r\in S$, we can  define a ring homomorphism $\calk[x_1,..,x_r,y,G^{-1}]\overset{\varphi}{\lrta} \overline{\calk}$ where $x_i\mapsto s_i$, $ y\mapsto$ any root of the minimal polynomial evaluated in $s_i$. $a_N(s_1,...,s_r)T^N+...$ and $G^{-1}\mapsto G(s_1,...,s_r)^{-1}$. since $G(s_1,...,s_r)\neq0$  also we have $a_N(s_1,....,s_r)\neq 0$ (The minimal polynomial of $y$ with coefficients evaluated in $s_i$ indeed has nontrivial roots in $\overline{\calk}$). The ring homomorphism is well-defined and $\calr\subseteq\calk[x_1,..,x_r,y,G^{-1}]$
$\varphi(x_1-s_s)=0$, where $(x_1-s_s)$ is invertible in $\calr$, which makes the contradiction.
\end{proof}


\begin{lemma}(Second half of Lemma~\ref{lem:PV})
$\calk$ is a differential field and $\calc$ is the field of constant, $\calc=\overline{\calc}$ and $char \calc=0$. $\calr/\calk$ simple differential ring which is finitely generated as $\calk$-algebra. $\Lrta $ the field of constants of $\calr$ is $\calc$.
\end{lemma}
\begin{proof} We already know $\calr$ is an integral domain.
Let $\call=Frac(\calr)$, fix $a\in \call, a\neq 0, a'=0$. Suppose $a\notin \calc$, consider the ideal $I:=\{b\in \calr|a\cdot b\in \calr\}\subseteq \calr$. This is a differential ideal because $b\in I\Lrta a b'=a'b+ab'=(ab)'\in \calr$. By the assumption $\calr$ is simple differential ring $\Lrta I=\calr$. Then $1\in I\Lrta a\cdot 1\in \calr$. $a$ has an inverse in $\call$, denote it by $c$. Then $e\neq 0,e'=0$ we can proceed the similar construction $J:=\{b\in \calr|e\cdot b\in \calr\}\subseteq \calr$ it also indicates that $e\in \calr$, hence we get the conclusion that $a\in \calr^\times$

Same argument for $a+c$ for any $c\in \calc$ shows 
$(a+c)\in \calr^\times,\forall c\in \calc\Lrta a$ is algebraic over $\calk$ $\overset{Exercise~\ref{exr:1.7}}{\Lrta} a $ is algebraic over $\calc=\overline{\calc}$ $\Lrta a\in \calc$
\end{proof}

\begin{prop} $\calk$ is a differential field with constants $\calc=\overline{\calc}$
Let $u'=A u$ be a matrix differential equation over $\calk$. 
\begin{enumerate}[label=(\arabic*)]
\item A Picard-Vessiot extension for $u'=A u$ exists. 
\item Any two P-V extension for $u'=Au$ are isomorphic.
\item The field of constant of any P-V extension is $\calc=\overline{\calc}$ 
\end{enumerate}
\end{prop}
\begin{proof}
The previous Lemma $\Lrta $(3)

For (1) consider the ring $\calr_0=\calk[(X_{ij})_{1\leq i,j\leq n},det(X)]$. Define a differentiation n $\calr_0$by 
$$
X'=AX
$$
$$
X_{ij}'= (AX)_{ij}\text{ a polynomial in} \in \calk[X_{ij},..,X_{nn}]
$$
and together with the Leibnitz rule it is a well-defined differentiation on $\calr_0$.

Pick any maximal differential ideal $I\subseteq \calr_0$ and set $\calr=\calr_0/I$. $\calr$ is a P-V ring:\\
Simple because $I$ is maximal.\\
Fundamental matrix of solutions is $X$ (the classes of $X$ in $\calr_0/I$)\\
$\calr$ is generated by $X_{ij}$ and $det(X)^{-1}$.

For (2) Let $\calr_1,\calr_2$ be P-V rings. Consider $\calr=\calr_1\otimes \calr_2$ with differential $(a\otimes b)'=a'\otimes b+a\otimes b'$. Choose $I\subseteq \calr$ maximal differential ideal.
Consider $\varphi_1:\calr_1\lrta \calr/T|\varphi_1(a)=a\otimes 1$ and $\varphi_2:\calr_2\lrta \calr/T|\varphi_2(1)=1\otimes b$. $\varphi_1$ and $\varphi_2$ are morphism of differential rings and since $\calr_1,\calr_2$ are simple, $\varphi_1,\varphi_2$ are injective. Let $V_1\in M_n(\calr_1), V_2=M_n(\calr_2)$ be fundamental matrices of solution of $u'=Au$. $\varphi_1(V_1)$ and $\varphi_2(V_2)$ are fundamental matrices of solution in $\calr/I$. $\calr/I$ is simple finitely generated $\Lrta$ constants in $\calr/I$ are $\calc$.
$\exists S\in GL_{n}(\calc)$ with $\varphi_1(V_1)=\varphi_2(V_2) S$

$\varphi_1(\calr_1)$ is isomorphic to the algebra in $\calr/I$ generated by $\varphi_1(V_{1,ij})$ and $\varphi_1(det(V_1))^{-1}=$ the algebra in $\calr/I$ generated by $\varphi_2(V_{2,ij})$ and $\varphi_2(det(V_2))^{-1}$ $\cong \varphi_2(\calr_2)$

Then $\calr_1\cong \varphi_1(\calr_1)=\varphi_2(\calr_2)\cong \calr_2$
\end{proof}

\section{The Differential Galois Groups}

Assumption: $\calk$- differential field with $char 0$, $\calc$ is the set of constants in $\calk$ and $\calc=\overline{\calc}$.
\begin{dfn}
Let $\calr$ be a Picard-Vessiot ring of a differential equation $u'=Au$ or of a differential module $(M,\pd)$ over $\calk$. We call \textbf{Galois group of the equation/ module} the group 
$Aut^{\pd}(\calr/\calk)=$\{$\calk$-algebra isomorphism $\varphi:\calr\lrta \calr$ compatible with the differentiations\}. Usually we denote it with $\text{Gal}^{\pd}(\calr/\calk)$
\end{dfn}

Exercise: Let $\call/\calk$ be a finite Galois extension.
\begin{enumerate}[label=(\arabic*)]
\item There is unique differentiation on $\call$ extending that of $\calk$.
\item Look at $\call$ as a $\calk$-module (differential module), Then a Picard-Vessiot extension fo $\call$ is $\call$ as a $\calk$-algebra.
\item $\text{Gal}^{\pd}(\call/\calk)=\text{Gal}(\call/\calk)$
\end{enumerate}


$\text{Gal}^\pd(\calr/\calk)$ can be seen as a subgroup of $GL_n(\calc)$
Let $V\in GL_n(\calr)$ be a fundamental matrix of solutions. Pick $g\in G=\text{Gal}^\pd(\calr/\calk)$. then  $gV=g(v_{ij})$ is again a fundamental matrix of solutions.
$$
(gV)'=g V'=g AV=A(gV)
$$
$$
g(V)=V\cdot \gamma(g)
$$
$\gamma\in GL_n(\calc)$ is unique, because two fundamental matrices are linked with a unique matrix in $GL_n(\calc)$ (Remark~\ref{rmk:fundamental_matrix}). Then we get a group homomorphism: 
$$
\begin{aligned}
\gamma:&G\hookrightarrow GL_n(\calc)\\
& g\longmapsto \gamma(g)
\end{aligned}
$$
It is injective:
$\gamma(g)=\mathbbm{1}\Lrta gV=V$, but $\calr$ is generated by entries of $V\Lrta g=id_\calr=\mathbbm{1}_G$.


What makes differential Galois groups a powerful tool is that they are linear algebraic groups and, moreover, establish a Galois correspondence, analogous to the classical Galois correspondence. Torsors will explain the connection between the Picard-Vessiot ring and the differential Galois group. The Tannakian approach to linear differential equations provides new insight and useful methods. Some of this is rather technical in nature. We will try to explain theorems and proofs on various levels of abstraction.

\begin{ex}
$\calk=\cplx(t),\calc=\cplx$
$u'=Au$
$$
A=\begin{pmatrix}
0 & 1\\
0 & -1/t
\end{pmatrix}
$$
$X:=\log(t)$. The Picard-Vessiot ring $\calr=\cplx(t)[X]$, with differential defined by $X'=\frac{1}{t}$+Leibnitz.
$\text{Aut}^\pd(\calr/\calk)\ni g$, the action of $g$ is $\cplx(t)$-linear.
$$
\begin{aligned}
g:&\cplx(t)[X]\Lrta \cplx(t)[X]\\
& X\longmapsto g(X)
\end{aligned}
$$
It is compatible with the differentiation
$$
\begin{aligned}
g(X)'&=g(X')=g(1/t)=1/t\\
g(X)&=X+a,\ a\in\cplx
\end{aligned}
$$
For a fundamental matrix
$$
V=
\begin{pmatrix*}
1 & X\\
0 & 1/t
\end{pmatrix*}
$$

$$
g(V)=
\begin{pmatrix*}
1 & X+a\\
0 & 1/t
\end{pmatrix*}
=\begin{pmatrix*}
1 & X\\
0 & 1/t
\end{pmatrix*}
\cdot 
\begin{pmatrix*}
1 & a\\
0 & 1
\end{pmatrix*}
$$
Then 
$$
\text{Gal}^\pd(\calr/\calk)
=\left\{\left.\begin{pmatrix*}
1 & a\\
0 & 1
\end{pmatrix*}
\right| a\in \cplx
\right\}
\cong(\cplx,+)
$$
\end{ex}







\end{document}
