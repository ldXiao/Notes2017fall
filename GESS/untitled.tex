\documentclass[11pt]{article}
\usepackage{amssymb}
\usepackage{latexsym}
\usepackage{amsmath}
\usepackage{amsthm}
\usepackage{mathtools}
\usepackage{natbib}
\usepackage{tikz-cd}
\usepackage{enumitem} 
\usepackage{hyperref}
\hypersetup{
    colorlinks,
    citecolor=black,
    filecolor=black,
    linkcolor=black,
    urlcolor=black
}
\newtheorem{thm}{Theorem}[section]
\newtheorem{prop}[thm]{Proposition}
\newtheorem{lemma}[thm]{Lemma}
\newtheorem{exercise}[thm]{Exercise}
\newtheorem{cor}[thm]{Corollary}
\newtheorem{dfn}[thm]{Definition}
\newtheorem{axiom}[thm]{Axiom}
\newtheorem{rmk}[thm]{Remark}
\newtheorem{ex}[thm]{Example}
\newtheorem{question}[thm]{Question}
\newtheorem{problem}[thm]{Problem}
\renewcommand{\baselinestretch}{1.05}
\newcommand{\pd}{\partial}
\newcommand{\reals}{\mathbb R}
\newcommand{\cplx}{\mathbb C}
\newcommand{\intg}{\mathbb Z}
\newcommand{\bbf}{\mathbb F}
\newcommand{\bbk}{\mathbb K}
\newcommand{\ratl}{\mathbb Q}
\newcommand{\torus}{\mathbb T}
\newcommand{\sca}{{\mathfrak a}}
\newcommand{\scb}{{\mathfrak b}}
\newcommand{\scc}{{\mathfrak c}}
\newcommand{\scm}{{\mathfrak m}}
\newcommand{\scn}{{\mathfrak n}}
\newcommand{\scp}{{\mathfrak p}}
\newcommand{\frakg}{{\mathfrak g}}
\newcommand{\frakd}{{\mathfrak d}}
\newcommand{\calf}{{\cal F}}
\newcommand{\calg}{{\cal G}}
\newcommand{\cala}{{\cal A}}
\newcommand{\calc}{{\cal C}}
\newcommand{\cale}{{\cal E}}
\newcommand{\calk}{{\cal K}}
\newcommand{\call}{{\cal L}}
\newcommand{\caln}{{\cal N}}
\newcommand{\calo}{{\cal O}}
\newcommand{\calr}{{\cal R}}
\newcommand{\mathbold}{\bf}
\newcommand{\cinf}{C^{\infty}}
\newcommand{\row}[2]{#1_1,\dots ,#1_{#2}}
\newcommand{\dbyd}[2]{{\pd #1\over\pd #2}}
\newcommand{\Space}{{\bf Space}}
\newcommand{\alg}{{\mathbold Alg}}
\newcommand{\notsubset}{\not \subset}
\newcommand{\notsupset}{\not \supset}
\newcommand{\pois}{{\mathbold Pois}}
\newcommand{\pitilde}{\tilde{\pi}}
\renewcommand{\qedsymbol}{$\square$}
\bibliographystyle{plain}
\title{\bf Questions for Thurston/Blor}
\author{by Lin-Da Xiao} %\thanks{Research partially supported by NSF Grant DMS-96-25122 and the Miller Institute for Basic Research in Science.}
\begin{document}
\maketitle
At page 2 of Thurston's expository article on the essence of mathematical proofs. He gave a recursive definition of mathematics.
\begin{itemize}
\item Mathematics includes the natural numbers and plane and solid geometry.
\item Mathematics that which mathematicians study.
\item Mathematicians are those humans who advance human understanding of mathematics.  
\end{itemize}  
How to understand this? Obviously, this definition renders mathematics a time-dependent concept. If none of the mathematicians is studying some undiscovered subfield of mathematics today, this subfield is not part of math, right? 

On the other hand, we can make similar definition which is less dependent on the details of \textbf{human understandings}:
\begin{itemize}
\item Mathematics includes the natural numbers and plane and solid geometry.
\item Mathematics that which mathematicians study.
\item Mathematicians are those humans who extend to realm of mathematics.  
\end{itemize}
Then, the question is \textbf{would mathematics be an infinite set under certain isomorphism?}, Is this definition independent of the universe we are in? (For example, we can consider a universe of only one integer $0$, the $0$ him/herself is a mathematician).

For the article, I would say his argument involving $\intg/3\intg$ does not make much sense way we can't define $2+2=4$ in it. On the other hand  in the category of rings, we can say that $2+2=4$ is a \textbf{universal identity}, which means we can always find a unique ring morphism $\varphi:\intg\rightarrow R$ which maps $2+_\intg2=_\intg4$ to $\varphi(2)+_R\varphi(2)=_R\varphi(4)$, where $+_R$ and $=_R$ is the addition and equality in the ring $R$, this is true because $\intg$ is the initial object in the category of rings.
\end{document}
