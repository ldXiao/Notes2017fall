\documentclass[11pt]{article}
\usepackage{amssymb}
\usepackage{latexsym}
\usepackage{amsmath}
\usepackage{amsthm}
\usepackage{mathtools}
\usepackage{natbib}
\usepackage{tikz-cd}
\usepackage{enumitem}
\usepackage{hyperref}
\hypersetup{
    colorlinks,
    citecolor=blue,
    filecolor=red,
    linkcolor=blue,
    urlcolor=blue
}
\newtheorem{thm}{Theorem}[section]
\newtheorem{prop}[thm]{Proposition}
\newtheorem{lemma}[thm]{Lemma}
\newtheorem{exercise}[thm]{Exercise}
\newtheorem{cor}[thm]{Corollary}
\newtheorem{dfn}[thm]{Definition}
\newtheorem{axiom}[thm]{Axiom}
\newtheorem{rmk}[thm]{Remark}
\newtheorem{ex}[thm]{Example}
\newtheorem{question}[thm]{Question}
\newtheorem{problem}[thm]{Problem}
\renewcommand{\baselinestretch}{1.05}
\newcommand{\pd}{\partial}
\newcommand{\reals}{\mathbb R}
\newcommand{\cplx}{\mathbb C}
\newcommand{\intg}{\mathbb Z}
\newcommand{\bbf}{\mathbb F}
\newcommand{\bbk}{\mathbb K}
\newcommand{\bbp}{\mathbb P}
\newcommand{\ratl}{\mathbb Q}
\newcommand{\torus}{\mathbb T}
\newcommand{\sca}{{\mathfrak a}}
\newcommand{\scb}{{\mathfrak b}}
\newcommand{\scc}{{\mathfrak c}}
\newcommand{\scm}{{\mathfrak m}}
\newcommand{\scn}{{\mathfrak n}}
\newcommand{\scp}{{\mathfrak p}}
\newcommand{\frakg}{{\mathfrak g}}
\newcommand{\frakd}{{\mathfrak d}}
\newcommand{\calf}{{\cal F}}
\newcommand{\calg}{{\cal G}}
\newcommand{\cala}{{\cal A}}
\newcommand{\calc}{{\cal C}}
\newcommand{\cale}{{\cal E}}
\newcommand{\calk}{{\cal K}}
\newcommand{\call}{{\cal L}}
\newcommand{\caln}{{\cal N}}
\newcommand{\calo}{{\cal O}}
\newcommand{\calr}{{\cal R}}
\newcommand{\mathbold}{\bf}
\newcommand{\cinf}{C^{\infty}}
\newcommand{\row}[2]{#1_1,\dots ,#1_{#2}}
\newcommand{\dbyd}[2]{{\pd #1\over\pd #2}}
\newcommand{\Space}{{\bf Space}}
\newcommand{\alg}{{\mathbold Alg}}
\newcommand{\notsubset}{\not \subset}
\newcommand{\notsupset}{\not \supset}
\newcommand{\pois}{{\mathbold Pois}}
\newcommand{\pitilde}{\tilde{\pi}}
\renewcommand{\qedsymbol}{$\square$}
\newcommand{\rta}{\rightarrow}
\newcommand{\Lrta}{\Longrightarrow}
\newcommand{\lrta}{\longrightarrow}
\newcommand{\Llta}{\Longleftarrow}
\newcommand{\lgl}{\langle}
\newcommand{\rgl}{\rangle}
\newcommand{\inj}{\hookrightarrow}
\bibliographystyle{plain}
\title{\bf Cosmic Galois Group}
\author{Texed by Lin-Da Xiao} %\thanks{Research partially supported by NSF Grant DMS-96-25122 and the Miller Institute for Basic Research in Science.}
\begin{document}
\maketitle
\tableofcontents
\newpage
\section{Statement of main results}
Marcolli and Connes defined it
\begin{thm}
For any Feynman graph $G$ with generic kinematics $q,m$ there is a canonical way to associate to a \textbf{convergent integral}
\begin{itemize}
\item an object $\text{mot}_G$ in $\mathcal{H}(S)$, where $S$ is a Zarikski open in a space of kinematics
\item .....
\item ....
\end{itemize}
\end{thm}
\section{Feynman graph and graph polynomials}
A Feynman graph is a graph $G$ defined by
$(V_G,E_G,E_G^{ext}),$ where $V_G$ is the set of vertices of $G$, $E_G$ is the set of internal edges of $G$, and $E^{ext}_G$ is a set of external legs. Their endpoints are encoded by the maps $\pd:E_G\lrta Sym^{2}V_G$ and $\pd:E_G^{ext}\lrta V_G$. We shall assume that the vertices with external legs lie in a single connected  component of $G$. A Feynman graph additionally comes with kinematic data:
\begin{itemize}
\item a particle mass $m_e\in \reals$ for every internal edge $e\in E_G$.
\item a momentum $q_i\in \reals^d$ for every external leg $i\in E^{ext}_G$,
\end{itemize}
where $d\geq 0$ is the dimension of space-time. All the external legs will be oriented inwards, so all momenta are incoming and are subject to momentum conservation.

In this paper, a subgraph $H$ of $G$ will be graph defined by a triple $(V_H,E_H,E^{ext}_H)$ where $V_H\subset V_G$, $E_H\subset E_G$ and either $E^{ext}_H=E^{ext}_G$ or $E^{ext}_H=\emptyset$.

A tadpole is a subgraph of the the form $\{\{v\},\{v,v\},\emptyset\}$. We shall use the following notation for the basic combinatorial invariants of $G$:
\begin{itemize}
\item $h_G=dim(H^1(G))$ the loop number of $G$
\item $\kappa_G=dim(H^0(G))$ the number of connected components of $G$
\item $N_G=|E_G|$ the number of connected components of $G$.
\end{itemize}
They do not depends on the external legs of $G$. Euler's formula states that
$$
N_G-V_G=h_G-\kappa_G.
$$
We define that If a vertex $v\in V_G$ has several incoming momenta $q_1,...,q_n$ we can replace it with a single incoming momentum $q_1 +...+ q_n$. Our notion of Feynman subgraph respects this equivalence relation. Then the graph polynomial defined latter would only depend on the equivalence classes.

We say that a Feynman graph is \textbf{of type $(Q,M)$} is it is equivalent to a graph with at most $Q$ external kinematic parameters and at most $M$ nonzero particle mass. We shall call a graph one-particle irreducible, or 1PI, if every connected component is 2-edge connected (i.e. deleting any edge causes the loop number to drop).
\subsection{Graph polynomials}
Let $G$ be a Feynman graph. Recall that a tree is a connected graph $T$ with $h_T=0$. A forest is any graph with $h_T=0$.
\begin{dfn}
Let $G$ be a connected Feynman graph. The \textbf{Kirchhoff polynomial} (or first Symanzik polynomal) is the polynomial in $\intg[\alpha_e,e\in E_G]$ defined by
\begin{equation}\label{eq:pol_Psi}
\Psi_G=\sum_{T\subset G}\prod_{e\not \in T}\alpha_e,
\end{equation}
where the sum is over all spanning trees $T$ of $G$. If $G$ has several connected components $G_1,...,G_n$, we shall defined
$$
\Psi_G=\prod_{1}^n\Psi_{G_i}
$$
The $second Symanzik polynomial$ is defined for connected $G$ by
\begin{equation}\label{eq:pol_Phi}
\Phi_G(q)=\sum_{T_1\cup T_2\subset G}(q^{T_1})^2 \prod_{e\not \in T_1\cup T_2}\alpha_e,
\end{equation}
where the sum  is over all spanning 2-trees $T=T_1\cup T_2$ of $G$, and $q^{T_1}:=\sum_{i\in E^{ext}_{T_1}}q_i$ is the total momentum entering $T_1$.
\end{dfn}

\begin{rmk}
$\alpha_e$ are just the Schwinger parameters
\end{rmk}

\begin{dfn}
Let $G$ be a Feynman graph. Define
$$
\Xi_G(q,m)=\Phi_G(q)+\left(\sum_{e\in E_G}m_e^2\alpha_e\right)\Psi_G.
$$
It is the denominator of Feynman integral, and it is homogeneous in $\alpha_e$ of degree $h_G+1$
\end{dfn}
Since the graph polynomials only depend on the total momentum flow, they are well-defined on equivalence classes of graphs.
\subsection{Feynman integral in projective space}
After omitting certain pre-factors, we define the Feynman integral
$$
I_G(q,m)=\int_\sigma \omega_G(q,m),
$$
where
$$
\omega_G(q,m)=\frac{1}{\Psi_G^{d/2}}\left(\frac{\Psi_G}{\Xi_G(q,m)}\right)^{N_G-h_Gd/2}\Omega_G
$$
and
$$
\Omega_G=\sum^{N_G}_{i=1}(-1)^i\alpha_i d\alpha_1\wedge...\wedge \widehat{d\alpha_i}\wedge...\wedge d\alpha_{N_G}
$$
Following form the fact that $deg(\Psi_G)=h_G$ and $deg(\Xi_G)=h_G+1$, we know that $\omega_G$ is homogeneous of degree $0$.

Finally, let $\sigma\subset \bbp^{N_G-1}(\reals)$ be the coordinate simplex defined in projective coordinates by
$$
\sigma=\{(\alpha_1:...:\alpha_{N_G})\in \bbp^{N_G-1}(\reals):\alpha_i\geq0\}.
$$
\subsection{Edge subgraphs and their quotients}

Let $G=(V_G,E_G,E_G^{ext})$ be a Feynman graph. A set of internal edges $\gamma\subset E_G$ defines a subgraph of $G$ as follows. Write $E_\gamma=\gamma$ and let $V_\gamma$ be the set of endpoints of elements of $E_\gamma$.
\begin{dfn}
A set of edges $\gamma\subset E_G$ is \textbf{momentum-spanning} if $\partial E^{ext}_G\subset V_\gamma$, and the vertices $E^{ext}_G$ lie in a single connected component of the graph $(V_\gamma, E_\gamma)$.
\end{dfn}

we define the subgraph associated to $\gamma\subset E_G$ by
$$
(V_\gamma, E_\gamma, E^{ext}_\gamma),
$$
where $E^{ext}_\gamma=E^{ext}_G$ if $\gamma$ is momentum-spanning and $E_\gamma^{ext}=\emptyset$ otherwise. We all $(V_\gamma, E_\gamma, E^{ext}_\gamma)$ the edge-subgraph associated to $\gamma$ and denote it also by $\gamma$ when no confusion arises.

The quotient of $G$ by an edge-subgraph $\gamma$ is defined by
$$
G/\gamma=(V_G/\sim,(E_Q\backslash \gamma)/\sim, {E_G}^{ext}/\sim),
$$
where $\sim$ is the equivalence relation on vertices of $G$ where two vertives are equivalent if and only if they are vertices of the same connected component of $\gamma$, and the induced equivalence relation on edges (unordered pairs of vertices). It is a Feynman graph. Every connected components of $\gamma$ corresponds to a unique vertex in $G/\gamma$. Note that $\gamma$ is momentum-spanning if and only if $G/\gamma$ is equivalent to a graph with no external momenta. (which amounts to compress each component of $\gamma$ to a single vertex.).

In this way, exactly one of the two Feynman graphs $\gamma$ and $G/\gamma$ is equivalent to a Feynman graph with non-zero external momenta: if  is momentum spanning it is $\gamma$, otherwise it is $G/\gamma$.

\subsection{Contraction-deletion}
Let $G=(V_G,E_G,E^{ext}_G)$ be a Feynman graph. The deletion fo an edge $e$ in $G$ is the graph $G/e$ defined by deleting the edge $e$ but retaining its endpoints:
$$
G/e=(V_G,E_G\backslash\{e\}, E^{ext}_G).
$$
In general, it is not a union of Feynman graphs since momentum conservation may not hold on each of its connected components.

One sometimes encounters the following variant of the previous notion of graph-quotient. It will be denoted by a double slash to distinguish it from the ordinary quotient. For an edge-subgraph $\gamma$, let $G//\gamma$ be the empty graph if $h_\gamma>0$ and
$$
G//\gamma=G/\gamma
$$
if $\gamma$ is a forest. In the case of a single edge $e$, $G/e$ is empty whenever $e$ is a tadpole.

It follows from Euler's formula that
$h_G=h_\gamma+h_{G/\gamma}$
for any edge-subgraph $\gamma\subset G$ (which is not necessarily connected).
\begin{lemma}
(Contraction-deletion) Let $G$ be connected and $e\in E_G$. Then
$$
\Psi_G=\Psi_{G\backslash e}^0\alpha_e+\Psi_{G//e},
$$
$$
\Phi_G(q)=\Phi^0_{G\backslash e}(q)\alpha_e+\Phi_{G//e}(q),
$$
where $\Psi_{G\backslash e}^0$ is given by the right hand side of Eq~\ref{eq:pol_Psi}: it is $\Psi_{G\backslash e}$ if $G\backslash e$ is connected and $0$ otherwise. Likewise $\Phi_{G\backslash e}^0(q)$ is given by the right-hand side of of Eq~\ref{eq:pol_Phi}: it equals to $\Phi_{G\backslash e}(q)$ if $G\backslash e$ is connected and equals to $\Psi_{G_1}\Psi_{G_2}(q^{G_1})^2)=\Psi_{G_1}\Psi_{G_2}(q^{G_2})^2)$ if $G\backslash e$ has two connected components $G_1, G_2$.
\begin{proof}


AAAAAAAA
Let $T$ be a spanning $k$-tree of $G$. The edge $e$ is not an edge of $T$ if and only if $T$ is a spanning $k$-tree of $G\backslash e$. By the definition of graph polynomials, this gives rise to the first terms in the right-hand side of above equations in the lemma. Note that if $e$ is a tadpole, this is the only case which can ovvur. Now suppose that $e$ is not a tadpole. If $e$ is an edge of $T$, the $T/e$ is a spanning $k$-tree of $G\backslash e$. Conversely, if $T'$ is a spanning $k$-tree of $G/e$, then there is a unique component of $T'$ which meets the vertex in $G/e$. It follows that the inverse image of $T'$ in $G$ with the edge $e$, is a spanning $k$-tree in $G$. This establishes a bijection between the set of spanning $k$-trees in $T$ which contain $e$ and those $G/e$. The rest just follows from the definitoin of graph polynomials.
\end{proof} 
\end{lemma}










\end{document}
